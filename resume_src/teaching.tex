%%%%%%%%%%%%%%%%%%%%%%%%%%%%%%%%
%%%
%%% Research Summary
%%%
%%% Author - Steve Hurder
%%%
%%% Date Started: October 12, 2009
%%% Date Completed: November 15 , 2009
%%%%%%%%%%%%%%%%%%%%%%%%%%%%%%%%

\documentclass[11pt]{amsart}
% Copyright (c) 2012 Cies Breijs
%
% The MIT License
%
% Permission is hereby granted, free of charge, to any person obtaining a copy
% of this software and associated documentation files (the "Software"), to deal
% in the Software without restriction, including without limitation the rights
% to use, copy, modify, merge, publish, distribute, sublicense, and/or sell
% copies of the Software, and to permit persons to whom the Software is
% furnished to do so, subject to the following conditions:
%
% The above copyright notice and this permission notice shall be included in
% all copies or substantial portions of the Software.
%
% THE SOFTWARE IS PROVIDED "AS IS", WITHOUT WARRANTY OF ANY KIND, EXPRESS OR
% IMPLIED, INCLUDING BUT NOT LIMITED TO THE WARRANTIES OF MERCHANTABILITY,
% FITNESS FOR A PARTICULAR PURPOSE AND NONINFRINGEMENT. IN NO EVENT SHALL THE
% AUTHORS OR COPYRIGHT HOLDERS BE LIABLE FOR ANY CLAIM, DAMAGES OR OTHER
% LIABILITY, WHETHER IN AN ACTION OF CONTRACT, TORT OR OTHERWISE, ARISING FROM,
% OUT OF OR IN CONNECTION WITH THE SOFTWARE OR THE USE OR OTHER DEALINGS IN THE
% SOFTWARE.

%%% LOAD AND SETUP PACKAGES

\usepackage[margin=0.5in]{geometry} % Adjusts the margins

\usepackage{multicol} % Required for multiple columns of text

\usepackage{mdwlist} % Required to fine tune lists with a inline headings and indented content

\usepackage{relsize} % Required for the \textscale command for custom small caps text

\usepackage[pdftex]{hyperref} % Required for customizing links
\usepackage[dvipsnames]{xcolor} % Required for specifying custom colors
\definecolor{dark-blue}{rgb}{0.15,0.15,0.4} % Defines the dark blue color used for links
\hypersetup{colorlinks,linkcolor={dark-blue},citecolor={dark-blue},urlcolor={dark-blue}} % Assigns the dark blue color to all links in the template

\usepackage{tgpagella} % Use the TeX Gyre Pagella font throughout the document
\usepackage[T1]{fontenc}
\usepackage{microtype} % Slightly tweaks character and word spacings for better typography

\pagestyle{empty} % Stop page numbering

%----------------------------------------------------------------------------------------
%	DEFINE STRUCTURAL COMMANDS
%----------------------------------------------------------------------------------------

\newenvironment{indentsection} % Defines the indentsection environment which indents text in sections titles
{\begin{list}{}{\setlength{\leftmargin}{\newparindent}\setlength{\parsep}{0pt}\setlength{\parskip}{0pt}\setlength{\itemsep}{0pt}\setlength{\topsep}{0pt}}}{\end{list}}

\newcommand*\maintitle[2]{\noindent{\LARGE \textbf{#1}}\ \ \ \emph{#2}\vspace{0.3em}} % Main title (name) with date of birth or subtitle

\newcommand*\roottitle[1]{\subsection*{#1}\vspace{-0.3em}\nopagebreak[4]} % Top level sections in the template

\newcommand{\headedsection}[3]{\nopagebreak[4]\begin{indentsection}\item[]\textscale{1.1}{#1}\hfill#2#3\end{indentsection}\nopagebreak[4]} % Section title used for a new employer

\newcommand{\headedsubsection}[3]{\nopagebreak[4]\begin{indentsection}\item[]\textbf{#1}\hfill\emph{#2}#3\end{indentsection}\nopagebreak[4]} % Section title used for a new position

\newcommand{\bodytext}[1]{\nopagebreak[4]\begin{indentsection}\item[]#1\end{indentsection}\pagebreak[2]} % Body text (indented)

\newcommand{\inlineheadsection}[2]{\begin{basedescript}{\setlength{\leftmargin}{\doubleparindent}}\item[\hspace{\newparindent}\textbf{#1}]#2\end{basedescript}\vspace{-1.7em}} % Section title where body text starts immediately after the title

\newcommand*\acr[1]{\textscale{.85}{#1}} % Custom acronyms command

\newcommand*\bull{\ \ \raisebox{-0.365em}[-1em][-1em]{\textscale{4}{$\cdot$}} \ } % Custom bullet point for separating content

\newlength{\newparindent} % It seems not to work when simply using \parindent...
\addtolength{\newparindent}{\parindent}

\newlength{\doubleparindent} % A double \parindent...
\addtolength{\doubleparindent}{\parindent}

\newcommand{\breakvspace}[1]{\pagebreak[2]\vspace{#1}\pagebreak[2]} % A custom vspace command with custom before and after spacing lengths
\newcommand{\nobreakvspace}[1]{\nopagebreak[4]\vspace{#1}\nopagebreak[4]} % A custom vspace command with custom before and after spacing lengths that do not break the page

\newcommand{\spacedhrule}[2]{\breakvspace{#1}\hrule\nobreakvspace{#2}} % Defines a horizontal line with some vertical space before and after it


\newcommand{\grant}[7]{

\headedsubsection{#1}{#2--#3 (#4)} { % title, start date, end date, source
	\inlineheadsection{Investigators:}{#5}
	\vspace{1.2em}
	\inlineheadsection{Award:}{\$#6 (Share: \$#7)}
	\vspace{1.2em}
}
}

\newcommand{\colabgrant}[8]{

\headedsubsection{#1}{#2--#3 (#4)} { % title, start date, end date, source
	\inlineheadsection{Investigators:}{#5}
	\vspace{1.2em}
	\inlineheadsection{Award:}{\$#6 (Share: \$#7)}
	\vspace{1.2em}
	\inlineheadsection{Collaboration:}{#8}
	\vspace{1.2em}
}
}

\usepackage{graphicx}
\usepackage{amssymb}
\usepackage{epstopdf}
\usepackage{mfirstuc}
\usepackage{mdwlist}
\usepackage{hyperref}

\newcommand{\abr}[1]{\textsc{#1}}

%\usepackage[top=1in, bottom=1in, left=1in, right=1in]{geometry}

\newif\ifumd\umdfalse
\newcommand{\umdtext}[2]{
\ifumd
#1 #2
\else
#2
\fi
}


\newcommand{\image}[2]{  \begin{center}
\includegraphics[width=.5\linewidth]{images/#1}
\end{center}
  }

\newcommand{\student}[1]{\vspace{.5cm}\fbox{\parbox{0.95\linewidth}{{\small #1}}}\vspace{.5cm}}
\newcommand{\newcite}[2]{\capitalisewords{#1} et al.~\cite{#1-#2}}

\begin{document}

 \title{Teaching Statement}

 \author{Jordan Boyd-Graber}
\address{University of Maryland}
\email{jbg@boydgraber.org}

\date{July 2022}

\maketitle

When the pandemic started, I was on sabbatical.  So I didn't need to
dramatically change the way I taught overnight, thank heavens.
%
Moreover, unlike many of my peers, I
already had a large library of recorded lectures because I've been
doing a flipped classroom for a decade.  Here's my very first lecture
from 2013.
%
Nevertheless, I still had to change the way I taught.
%
Let me tell you how and why it took me so long to realize it.

\image{first_video_lecture}{}

\section{Flipped Classroom}

If you haven't heard the term before, a flipped classroom is where you
record the lectures, students watch them before class, and then you
use the normal class time to answer questions, work through exercises,
or to talk about tricky issues on homeworks.
%
Following the
suggestions of \newcite{Zappe}{09}, I keep my videos short (around 30
minutes total, edited for concision and broken into 5--15 minute
chunks).

For example, when I talk about topic models, I have a video where I
explain the basics by myself.
%
Then I bring in an external expert to
go over some of the details, and then I show a video from a conference
presentation on a topic.
%
The technical quality of the videos have
improved considerably over the years: I've moved from just a screen
capture to a multiple camera setup with green screen.

Then in class, I answer any questions students have and then we go
through an exercise.
%
For instance, we walk through Gibbs sampling
works for a toy example.
%
Students calculate—by hand—the sampling
equation and go through the process to get the answer.

% A lot of teachers do not like the flipped classroom, and I can
% understand why.
%
% First, it's a lot of work, and it requires planning.
% When I co-taught a flipped classroom with an uninitiated professor, he
% ended up cursing my name for convincing him to take part.

\subsection{Building Bonds}

There's research to suggest that the flipped classroom can help
students engage and retain information~\cite{Zuber-16}, but that's not
the real reason that I do it.
%
I do it because it helps me get to know the students better.  Computer
science is the biggest major at Maryland (over three thousand undergrads), and
\abr{ai}-adjacent fields are even more popular within that major, so my
classes are getting pretty darn big.
%
If I didn't do a flipped
classroom, I would know so many fewer students.

And the flipped classroom also helps form bonds between students.
%
When I was an undergrad, I wanted to take organic chemistry even
though I was a history / \abr{cs} major.
%
That failed because studying for
that class was pretty social, and I was not a chemistry major, so I
didn't have an existing support network coming in.
%
Now that I'm a
professor, one of the reasons I like the flipped classroom is that it
forces people to talk to each other to figure out what questions they
want to ask, work together to solve practice problems, etc.
%
Those
groups and those connections often carry over to homework groups and
project groups.

And this helps me serve the few iSchool, Linguistics, and other ``odd
ball'' students who, like Jordan in 2003 are trying to explore
something new to them and don't have an established buddy group (which
many of the CS students have).  The flipped classroom helps them get
to know the people who can help them survive and thrive in the course.

When I once I fell off the wagon and went back to a ``normal'' lecture
mode in 2016 because it was my first time teaching the class, it had a
hundred people, and I had no TA.  I was told that students coming in
would know Python.  But it turns out that a big contingent had only
played around with the language for a week or two and were unprepared!
Because I didn't flip the classroom, I found out when they were trying
to turn in the first homework that many students were really
struggling.  This wouldn't have happened in a flipped classroom where
I could see that the struggle was real in front of my own eyes.

So back to March 2020. To make sure that I did not miss out on
creating educational content, I started creating a weekly pub quiz to
replace quiz canceled by our usual haunt, closed of course by the
pandemic.  And as the pandemic dragged on and on, I learned how to
create interactive, fun environments that got small groups to discuss
problems, come to a solution, and then share it with the group---the
same thing that I'd need to move the interactive portion of hybrid
classes online.  And I
learned how to broadcast with a green screen in real-time and share
multimedia with the group.  (All outlined in \href{https://docs.google.com/document/d/1YesfpZ_-b2mT3BkTrlOLD5epZxz4uF1DqUkHZHrn344/edit?usp=sharing}{this document} that formed
the foundation for pandemic-era pub quizzes.)  I used the same
format and technology for \href{https://sites.google.com/view/qanta/past-events/neurips-2020-efficient-qa}{Google's Efficient QA competition} that I
hosted in December 2020.

\image{efficient_qa}{}

\section{Hybrid Classrooms}

Given all of this, I thought that when we (mostly) returned into
hybrid classrooms, I'd be all set for the new world of teaching.
After all, I had already mastered asynchronous videos, and I had
practiced online-only interactions thanks to our quizzes.  I was not
ready, however.  In a real classroom, I could tell when a student was
lost or disengaged.  This is not true when many students are on Zoom.
A hybrid classroom has a particularly pernicious failure mode: a
student doesn't watch the videos, joins online, gets lost in the
discussion, and then silently disengages.

To make matters worse, many of the improved learning outcomes in
flipped classrooms come from regular quizzes~\cite{tune-13}, which if
nothing else would have diagnosed students disengaging. In an attempt
to make classroom management easier and to lighten the burden on
students, I suspended quizzes.  Another mistake!

And this isn't transitory.  Although it emerged under unpleasant
circumstances, I think the hybrid classroom is going to be with us for
a while for good reasons.  Even after we returned to classrooms,
students were struggling with visa issues, health issues beyond COVID,
were attending conferences, were at home for a wedding, or were snowed
in.  It makes the classroom more equitable and flexible.

And like many faculty, I'm still developing best practice on how to do
this.  I encourage people to keep their cameras on and I make a big
deal of the 10\% participation grade.  This creates some
annoyances\dots for example, students asking questions for the sake of
hitting the participation quota rather than actually wanting
information, but it's a small price to pay.  I'm proud that after a
shaky start and uncooperative \abr{av}, I'm getting good engagement
from both in-person and Zoom students.

\section{Assignments}

But teaching is not just about lectures.  My courses typically have a
few small, practical assignments (typically three to five), a midterm,
and a course project.

The assignments give continuity to the class and allow students to
practice skills introduced in the class; I encourage students to work
together to solve homework problems.
%
For example, students put together a logistic regression classifier to
determine if an answer to a question is correct or not.  In my mind,
an ideal homework assignment has an easy to achieve initial goal but
leaves room for exploration.  Returning to the logistic regression
example, just implementing stochastic gradient descent challenges some
students (and they stop there), but for students who want to explore
further, they can try out different step sizes and updating schedules
for extra credit.

The midterm serves as a reality check for both my students and me.  I
design exams with five to six free response questions (of which
students must answer a subset) that synthesize disparate concepts from
the course in a problem context (e.g., for a machine learning course,
proving the \abr{vc} complexity of a simple hypothesis class).  Based
on the results of the midterm, I can identify students that might need
extra help or what areas I need to cover in more detail.

But because I work on applying social science methods—like item
response theory—-to machine learning datasets to identify easy,
difficult, or ambiguous examples, it's only right that I take my own
medicine to examine multiple choice questions and use these methods to
figure out how good my exam questions are.

\section{Projects}

I want to make sure that my students can actually use their skills
once they're done with the class.  Thus, I typically end courses with
a project.  It reinforces key concepts from the course, connects those
concepts to the rest of their curriculum and research, and often
serves as a launching point to things that are useful in the real
world.  For undergraduate classes, this is more directed: in one class
they build trivia-playing robots to take on former Jeopardy! Champions
or Victoria Groce from the American version of \textit{The Chase}.  For
graduate courses, projects are more open-ended.  Projects in my
courses have become a comedy troupe's website, have unearthed
previously unknown primary sources on local history, helped students
advance in the workplace, and resulted in academic publications.  As a
testament to the effectiveness of these relationships, after one
graduate course I taught (Computational Linguistics II, UMD CMSC 773),
I ended up publishing papers with four of the eight students.

The projects help individuals calibrate the course to their own needs
and abilities; because the project is directed toward things they care
about, the teams working on the projects typically stretch their
abilities more than I could through predefined assignments.

\section{Outreach}

And I like to think that my teaching style has helped learners outside
the University of Maryland make progress toward learning about the
interaction between science and society.  The videos on my YouTube
channel have been viewed a million times and my most popular videos
about fifty thousand times.  This has offered me new connections:
e.g., people who come up to me at conferences to say that they learned
variational inference or TV producers wanting to make a Human
vs. Computer game show.

A major part of my research is making machine learning accessible to
high school students.
%
My human-computer question answering exhibition matches have attracted
thousands of interested high school students in \abr{dc}, Chicago,
Dallas, Atlanta, and Seattle.
%
I've also served as a mentor as a part of Maryland's Bitcamp and
Technica hackathon programs on projects of women and underrepresented
minorities to build \abr{qa} systems.

\image{atlanta_hsnct}{}

Thus, I'm glad that I had to up my game because of the pandemic,
starting with an \textit{ad hoc} trivia game in mid March 2020.  I
think it's a good example of how with technology we can better serve
those who want to learn, whether they're our students or not.  And
while some innovations are a matter of necessity rather than of
careful deliberation and insight, we should still embrace them if they
help students learn better.

\section{Mentoring}

\subsection{Mentoring Undergraduates}

Like everything else, the pandemic has changed the way I work with
undergraduate researchers.
%
I began a ``anyone welcome'' summer virtual internship for
undergraduate students to work on small, bite-sized projects that fit
into a larger research program (usually directed by a mid-career grad
student).
%
Our work with undergraduates has been published in top venues like
\abr{naacl} and \abr{chi}, and the undergraduates I've worked with
have gone on to have successful research careers (e.g., Lester Mackey
became faculty at Stanford before moving to Microsoft Research, Eric
Wallace is now a PhD student at Berkeley).
%
My current star undergraduate student, Chenglei Si (class of 2024) already
has four publications in top venues.

\subsection{Mentoring Graduate Students}

I've graduated seventeen PhD students and am the chair or co-chair for
eight current students.  I like to have a group meeting every other
week with students I'm working with (broadly construed), and
one-on-one meetings as needed with students, typically once a week. In
addition, everyone in my group (me included) sends a weekly e-mail to
the group saying: what they worked on that week; what they plan to work
on next week; anything that's holding them up or blocking their
progress. I use an Internet chat program (Slack) to communicate with remote
students and for lower-latency conversations than e-mail.

In their first year, students typically work on a starter project that
builds on a senior student's work, this often becomes a paper with the
new student as a first author.  From there, I work with the student to
craft a trajectory of papers that will form a foundation for the rest
of their graduate studies.  While I'm fairly hands-on with conference
and journal submissions, I never edit proposals or dissertations; I
provide verbal feedback and suggestions but students must find their
own voice in the writing process.

After graduation, my students have gone on to good positions in
industry (e.g., Forough Poursabzi and Ahmed Elgohary at Microsoft
Research; Pedro Rodriguez at \abr{fair}) and academia (e.g., Alvin
Grissom II, Assistant Professor at Haverford; He He, Assistant
Professor at NYU; Mohit Iyyer, Assistant Professor at UMass Amherst).

\section{Example Sylabuses}

Part of my commitment to public education is to ensure that all of my
course materials (lectures, syllabus, assignments) are publicly
accessible. The most recent version of each of my courses is listed at
\url{http://users.umiacs.umd.edu/~jbg/static/courses.html}.  My
lectures, which have approximately a million views, are available on
YouTube at \url{https://www.youtube.com/c/JordanBoydGraber/videos}.

% \section{What Students Say \dots}

% Below are cherry-picked comments from students either sent as an unsolicited
% e-mail (anonymized and presented with permission) or taken from anonymous course
% evaluations.

% \student{ I just started a new job on Monday as a Librarian at [awesome place].
%   I was hired to catalog a special project with the archives and never thought
%   I'd need anything from [introduction to information technology] \dots And then
%   on the first day we started talking about the database of objects.  When we
%   finally got access to the database, I realized it was entirely a SQL database
%   and had to use everything you taught us last semester to help think of how to
%   run the queries and ask the designer what it is capable of.  }


% \student{
% I actually had a job interview for a professional librarian job this week and it
% was about digital libraries.  They asked me questions about what do I know about
% IP validation, HTML, servers, etc.  Thankfully, fresh out of your class, I was
% able to say ``actually, I know a TON about this stuff! There was a question on
% our midterm about IP validations and everything!''

% I wanted to write to let you know that the things I learned in your class this semester are already having a huge impact on my career.
% }

% \student{I showed the \dots app I made for my final project to my boss, and she
%   ran downstairs to [important person]. He came upstairs to see my app and to
%   talk to me about it. Turns out he has a team researching vendors who could
%   provide our library with a mobile presence much like the one I made for my
%   final project in your class. He was excited to see what I'm capable of and
%   will be putting me in touch with the people on the project. This is a big deal
%   for an entry-level librarian like me.}



%%%%%%%%%%%%%%%%%%%%%%%%%%%%%%%%

\clearpage

\bibliographystyle{resume_src/splncs03}

\bibliography{resume_src/journal-full,resume_src/jbg}
\noindent\rule{4cm}{0.4pt}
\end{document}

\end{document}
