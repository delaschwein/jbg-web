\documentclass[11pt,twoside]{article}
\usepackage{alltt}
\usepackage{comment}

\textwidth=6.5in
\oddsidemargin=0.25in
\evensidemargin=0.25in
\topmargin=-0.1in
\footskip=0.8in
\parindent=0.0cm
\parskip=0.3cm
\textheight=8.00in
\setcounter{tocdepth} {3}
\setcounter{secnumdepth} {2}
\sloppy

\newcounter{lecnum}

\newcommand{\answer}[1]{{\bf #1}}

\newcommand{\lecture}[5]{
   \pagestyle{myheadings} \thispagestyle{plain} \newpage
\setcounter{lecnum}{#1} \setcounter{page}{1} \noindent
\begin{center}
\framebox{\vbox{
{Homework #1 \hfill   #2} \\
\vspace{2mm}
{\bf Literature Review}\\
\vspace{2mm}
{Due: #3   \hfill #4}\\
\vspace{4mm}}}
\end{center}

\markboth{Homework #1: #3}{Homework #1: #3}\vspace*{4mm}}


\begin{document}

% Add your name in the last bracket if you use this template
\lecture{5}{COS/LIN 280}{December 12, 2008}{}

\section{Selecting a Paper}

Choose a paper that interests you; there is a list on Blackboard under ``Course Materials.''  If you want to review another paper, run it by either Dr. Fellbaum or Jordan (we'll almost certainly say yes).

\section{Evaluating the Paper}

Carefully review the paper in prose.  Make sure your writeup addresses the following issues:

\begin{enumerate}
\item What problem is the paper trying to address?
\item What solutions, if any, preceded this paper?
\item Given this background, what specifically has this approach added?
\item How did the authors evaluate the approach?
\item What data did the authors use?
\item How widely applicable is the method (either in terms of the size of the dataset or the scope of the theory)? 
\item What linguistic assumptions does the work make?  Are these assumptions justified?
\item How hard would it be for someone to reproduce the work?
\end{enumerate}

You should be able to cover these questions in about 1500 words.  More is fine, but concision should be one of your goals.

\end{document}