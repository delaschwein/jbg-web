\documentclass[11pt, amssymb, a4paper, one column]{article}
\usepackage{times, amsmath, amssymb, cancel, changepage, url, gensymb, graphicx, mathrsfs, amsthm, lipsum, fancyhdr}
\usepackage[margin=.75in]{geometry}
\usepackage{mdwlist}

\newcommand{\ihat}{\mathbf {\hat \imath}}
\newcommand{\jhat}{\mathbf {\hat \jmath}}
\newcommand{\abr}[1]{\textsc{#1}}


%you only really need a documentclass/ packages to make a document, the following is just nice formatting. You can easily remove it or look up something else
%that you like better

\pagestyle{fancy}
\lhead{Jordan Boyd-Graber, University of Maryland}
\rhead{\large Teaching Statement \normalsize}
%note how I use "\#" to denote "#." This is because # is reserved (for something). Similarly, % is reserved for comments, so to use it you need \%.
%other such things include { }.

\fancypagestyle{plain}{}

%end of header, beginning of document

\begin{document}


\paragraph{Interactive Classrooms}

I find traditional lecture dull, even when I'm the person in front of
the class.  Thus, I usually ``flip'' my classrooms to record
lectures so students can watch at home at their leisure.  This moves
discussion, working on homework, and student questions into class time
(where before they happened at home or not at all).  Although I've
been doing this since 2013, it has paid dividends after Corona by
keeping students engaged and maintaining flexibility.

For technical classes with a large hands-on component, I provide
students with ample opportunities to explore resources in a supportive
environment where I or their classmates---working together in small
groups---can help them overcome the small, unexpected hurdles that can
appear while exploring new concepts.


\paragraph{Evaluation and Course Activities}

I typically structure classes with a few small, practical assignments (typically
three to five), a midterm, and a course project.

The assignments give continuity to the class and reinforce
skills introduced in lecture.  For example, in a course introducing students to information
technology, they create a basic webpage; in a
class introducing students to cloud computing, they create an
algorithm to play the ``Kevin Bacon'' game (i.e., find the shortest path
between an actor and Kevin Bacon based on co-stars).

Finally, I use group projects to ensure students can apply what
they've learned in the course.  Projects in my courses have become a comedy
troupe's website, have unearthed previously unknown primary sources on
local history, helped students advance in the workplace, and resulted
in academic publications.  As a testament to the effectiveness of
these relationships, after one graduate course I taught (Computational
Linguistics II, UMD CMSC 773), I ended up publishing papers with half
of the enrolled students.


\paragraph{Mentoring Undergraduates}

Students realize that I'm passionate about research and that I'm also
approachable so that they can explore their interests. I've worked with nine
undergraduate students, our work with undergraduates has been published in top
venues like \abr{naacl} and \abr{chi}, and the undergraduates I've worked with
have gone on to have successful research careers (e.g., Lester Mackey became
faculty at Stanford before moving to Microsoft Research, Eric Wallace is now a
PhD student at Berkeley).

\paragraph{Mentoring Graduate Students}

I've graduated ten PhD students and am the chair or co-chair for eight
current students.  In their first year, students typically work on a
starter project that builds on a senior student's work, this often
becomes a paper with the new student as a first author.  From there, I
work with the student to craft a trajectory of papers that will form a
foundation for the rest of their graduate studies.

After graduation, my students have gone on to good positions in industry (e.g.,
Viet-An Nguyen, Facebook Data Science) and academia (e.g., He He, NYU faculty;
Mohit Iyyer, UMass faculty; Alvin Grissom II, Assistant Professor at
Haverford; Shudong Hao, Assistant Professor at Bard College).

\paragraph{Public Outreach}

I also strive to make research accessible to
high school students.  My human-computer question answering exhibition
matches have attracted thousands of interested high school students in
\abr{dc}, Chicago, Dallas, and Seattle.  Our research has been covered
by the Wall Street Journal, Newsweek, CNN, and Huffington Post.

\paragraph{Pedagogical development projects}

From 2011--2013, I chaired \abr{umd}'s Information Science
undergraduate education committee. We developed a new undergraduate
degree program to complement the university's existing strengths (and
exploding enrollments) in computer science. I developed the
proposal, articulation agreements with local community colleges,
planned course offerings, and shepherded the proposal through college
and provost approval.  Susan Winter and Vedat Diker (and many others)
took over the program, which became the university's fastest growing
major (958 majors in less than five years).

\paragraph{Example Sylabuses}

Part of my commitment to public education is to ensure that all of my course
materials (lectures, syllabus, assignments) are publicly accessible. The most
recent version of each of my courses is at
\url{http://users.umiacs.umd.edu/~jbg/static/courses.html}.  My
lectures, which have over 700 thousand views, are
on YouTube \url{https://www.youtube.com/c/JordanBoydGraber/videos}.

\end{document}
