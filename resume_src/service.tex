%%%%%%%%%%%%%%%%%%%%%%%%%%%%%%%%
%%%
%%% Research Summary
%%%
%%% Author - Steve Hurder
%%%
%%% Date Started: October 12, 2009
%%% Date Completed: November 15 , 2009
%%%%%%%%%%%%%%%%%%%%%%%%%%%%%%%%

\documentclass[11pt]{amsart}
\usepackage{graphicx}
\usepackage{amssymb}
\usepackage{epstopdf}
\usepackage{hyperref}



\usepackage[top=1in, bottom=1in, left=1in, right=1in]{geometry}

\newcommand{\student}[1]{\vspace{.5cm}\fbox{\parbox{0.95\linewidth}{{\small #1}}}\vspace{.5cm}}
\newcommand{\abr}[1]{\textsc{#1}}

\begin{document}

 \title{Statement on Academic Leadership and Service}

 \author{Jordan Boyd-Graber}
\address{University of Maryland}
\email{jbg@boydgraber.org}

\date{Updated July 2022}


\keywords{}

\maketitle

\section{Director of the Computational Linguistics Lab}

At the University of Maryland, I served as the director of the
Computational Linguistics and Information Processing (\abr{clip}) Lab
2018--2022 (with a pause in the middle while I was on sabbatical in
Z\"urich).
%
This required advocating for resources for the lab: money to host
visitors, space to seat students, and time from technical staff to
support our computational resources.
%
More importantly, it is also the director's responsibility to bring
the ten tenure-track faculty together for building consensus and shared
decision making on matters ranging from printer placement to student
discipline.

During this time, we had to substantially rebuild our computational
infrastructure.
%
When I took over as director, the lab was primarily using \abr{cpu}
nodes with data stored on \abr{zfs} on its own monolithic cluster.
%
As I'm wrapping up my term, we have moved to a cluster that's shared
with other research groups with high-memory \abr{gpu}s backed by a
modern distributed filesystem.
%
Making this transition required lobbying for funding, working with the
technical staff to implement the changes, coordinating equipment
purchases on individual grants, and then instructing students
on how to use the equipment as we navigate these transitions.

Apart from these computational changes, we also navigated the
pandemic's transitions from fully online to hybrid interactions:
exhorting students to follow the university's guidance---on everything
from social distancing and mask wearing---and supporting students and
faculty working from home to still participate in
lab life.

\section{Professional Service}

As an assistant professor I did the typical service roles: faculty advisor for
the 2015 \abr{acl} student research workshop, serving as area chair for
\abr{emnlp}, \abr{acl}, \abr{neurips}, \abr{icml}, and \abr{naacl}.

In 2022, after
\href{https://www.youtube.com/watch?v=3gSgNXGxzQU}{complaining about
  how hybrid conferences} created a ``separate but equal'' status for
virtual attendees, I was invited to be \abr{emnlp}'s first virtual
poster chair.
%
Our goal was use technology and prioritizing cross-modal interaction
to make sure that those unable to be in Abu Dhabi---either because of
visa status, health, or family obligations---are still able to
interact and be visible.

After accepting this position, I was asked to serve as a program chair for the
2023 meeting of the Association for Computational Linguistics, the top venue
for natural language processing research.
%
Our goal is to further bridge the divide between the online and in-person
participants: improving the online experience, focusing on interactions, and
lowering the costs for online participants.

% TODO: Corona, virtual poster session chair

% TODO: CLIP Director

\section{Creating a New Undergraduate Curriculum}

From 2011--2013, I chaired \abr{umd}'s Information Science
undergraduate education committee. During this time, we developed a
new undergraduate degree program to complement the university's
existing strengths (and exploding enrollments) in computer science. I
developed the initial proposal, articulation agreements with local
community colleges, planned course offerings, and shepherded the
proposal through college, provost, and State of Maryland Higher
Education Commission review.
%
Susan Winter and Vedat Diker (along with many others) took over the actual
implementation of the program, which became the university's fastest growing
major (958 majors in less than five years of existence).

While I'm sure those later steps were more difficult, my first steps
were not easy either.
%
I had to convince elements within the college that an
undergraduate program was indeed a good idea and would not distract or
detract from the College's existing (graduate-only) programs that were
already stable, successful (both academically and financially), and
well understood.
%
I crafted a data-driven message (using a topic model analysis) that
convinced faculty hold-outs and the administration that a new
undergraduate program that would be well-received by potential
students and employers.

We then created a set of concentrations, courses, and schedules to
offer the courses for initial cohorts.  We drafted course
descriptions, learning outcomes, and worked with feeder programs to
define prerequisites.

\section{Building the Faculty}

Apart from the undergraduate program, the largest investment of my
time in university service is in recruiting new faculty.
%
At Colorado, I served on the search committee that 
hired Jed Brown, Raf Frongillo, and Matt Hammer and lead the
recruitment of Raf Frongillo.
%
The following year, Martha Palmer and I successfully recruited Chenhao
Tan.

At Maryland, I served two years on a search committee for a joint hire
between the iSchool and Journalism (the first year failed) that hired
Naeemul Hassan.
%
Within computer science, I was a part of the search committee that
hired Rachel Rudinger, and I led her recruitment.

Because of my joint appointment across four units at Maryland, I have
served as mentor for multiple faculty: Yla Tausczik, Rachel Rudinger,
Babak Fotouhi, and Naeemul Hassan.

% Chenhao Tan

\section{Outreach to Undergraduates and High Schools}

% Mentoring the academic quiz team

My research is---in part---about human vs. computer question answering
competitions, so it's only natural that I also support our
undergraduate students who compete against other schools in academic
competitions by answering questions.

At the University of Colorado, this required building an organization
from the group up: the University of Colorado had previously not
competed in these events.
%
I organized three local high school tournaments to raise funds for so that
students could have the funds to travel to national competitions, which
they did for the first time in 2016 (after winning their region).
%
The accomplishment from my time at Colorado that I'm most proud of is
building this group of passionate undergraduates who developed both
competition-specific skills and more general leadership skills.

At Maryland, I advise an existing academic team that
consistently is in the top ten nationally.
%
I help advise the club leadership, provide institutional memory, and
connect the club with external organizations (e.g., setting up
cross-university scrimmages between \abr{umd} and Howard).

%\section{Looking Forward}



% TODO(UMD): Future work, language science

%%%%%%%%%%%%%%%%%%%%%%%%%%%%%%%%

%\bibliographystyle{../style/acl}
%\bibliography{../bib/journal-full,../bib/jbg}

\end{document}
