%%%%%%%%%%%%%%%%%%%%%%%%%%%%%%%%
%%%
%%% Research Summary
%%%
%%% Author - Steve Hurder
%%%
%%% Date Started: October 12, 2009
%%% Date Completed: November 15 , 2009
%%%%%%%%%%%%%%%%%%%%%%%%%%%%%%%%

\documentclass[11pt]{amsart}
\usepackage{graphicx}
\usepackage{amssymb}
\usepackage{epstopdf}


\usepackage[top=1in, bottom=1in, left=1in, right=1in]{geometry}

\newcommand{\student}[1]{\vspace{.5cm}\fbox{\parbox{0.95\linewidth}{{\small #1}}}\vspace{.5cm}}
\newcommand{\abr}[1]{\textsc{#1}}

\begin{document}

 \title{Teaching Statement}

 \author{Jordan Boyd-Graber}
\address{University of Colorado}
\email{jbg@boydgraber.org}

\date{Fall 2016}


\keywords{}

\maketitle

Teaching is one of the primary reasons I am in academia; in particular, I want
to excite students in science and technology. Here I describe how
participation and course activities help build this excitement in students and
how these methods help students build a career.

\section{Interactive Classrooms}

I find traditional lecture dull, even when I'm the person in front of
the class.  Thus, I usually ``flip'' my classrooms to record
lectures---with the help of the wonderful staff at Be Boulder
Anywhere---so students can watch at home at their leisure.  This moves
discussion, working on homework, and student questions into class time
(where before they happened at home or not at all).

Beyond structuring class to allow for many questions and discussion
points, I end most classes with a difficult discussion of a key
concept discussed in class and encouraging the class to work
collaboratively together to arrive at a solution.  For example:
\begin{itemize}
  \item After introducing relational databases, I work together with the class to
design a database scheme to serve the needs of a library circulation
system.  We talk through suggestions on what data should be stored,
how it should be represented, and what implications those choices
have.
  \item When teaching classifiers, students walk through the classification
    algorithms on tiny datasets (for example, documents with one or two words).
  \item When teaching annotation frameworks (called coding guides in the
    social sciences), I ask students to annotate data and
    compute their inter-annotator agreement.  I have discovered that this, more
    than any collections of examples I can provide, effectively convinces
    students the importance of having good input data for their algorithms.
\end{itemize}

For technical classes with a large hands-on component, I provide
students with ample opportunities to explore resources in a supportive
environment where I or their classmates---working together in small
groups---can help them overcome the small, unexpected hurdles that can
appear while exploring new concepts.

I also strive to encourage interaction outside of the classroom.  Every class I
teach uses a virtual space (Piazza) to encourage discussion and mutual support.  This
helps the entire class get answers to common questions, and it also serves as a
catalyst for spontaneous, unexpected communications: students sharing useful
resources with each other, organizing study groups, or sharing sci-fi books that
illustrate concepts.

I also try to be accessible to students through a variety of methods (e.g.,
Piazza, e-mail, office hours) to quickly answer questions and address their
concerns so that they do not get frustrated and so that they can make progress.

\section{Evaluation and Course Activities}

I typically structure classes with a few small, practical assignments (typically
three to five), a midterm, and a course project.

The assignments give continuity to the class and allow students to practice
skills introduced in the class; I encourage students to work together to solve
homework problems.  For example, in a course introducing students to information
technology, one assignment is to create a basic webpage; for a more advanced
class introducing students to cloud computing, one assignment is to create an
algorithm to play the ``Kevin Bacon'' game (i.e., to find the shortest path
between an actor and Kevin Bacon based on co-starring roles).

The midterm serves as a reality check for both my students and me.  I design
exams with five to six questions (of which students must answer a subset) that
synthesize disparate concepts from the course in a problem context (e.g., for a
machine learning course, creating a set of features for a new classification
problem).  Based on the results of the midterm, I can identify students that
might need extra help or what areas I need cover in more detail.

Finally, I use group projects to ensure students can apply what
they've learned in the course.  It reinforces key concepts from the
course, connects those concepts to the rest of their curriculum and
research, and often serves as a launching point to things that are
useful in the real world.  Projects in my courses have become a comedy
troupe's website, have unearthed previously unknown primary sources on
local history, helped students advance in the workplace, and resulted
in academic publications.  As a testament to the effectiveness of
these relationships, after one graduate course I taught (Computational
Linguistics II, UMD CMSC 773), I ended up publishing papers with four
of the eight students.

The projects help individuals calibrate the course to their own needs
and abilities; because the project is directed to things they
care about, the teams working on the projects typically stretch their
abilities more than I could through predefined assignments.  It also
helps build the close connection with the subject area, incubating a
comfort with science and technology.

\section{Mentoring Undergraduates}

Students realize that I'm passionate about research and that I'm also
approachable so that they can explore their interests.  I've worked
with nine undergraduate students, our work with undergraduates has
been published in top venues like \abr{naacl} and \abr{chi}, and the
undergraduates I've worked with have gone on to have successful
research careers (e.g., Lester Mackey is now faculty at Stanford).

\section{Mentoring Graduate Students}

I've graduated four PhD students and am the chair or co-chair for
seven current students.  I like to have a group meeting every other
week with students I'm working with (broadly construed), and
one-on-one meetings as needed with students, typically once a week. In
addition, everyone in my group (me included) sends a weekly e-mail to
the group saying: what they worked on that week; what they plan to work
on next week; anything that's holding them up or blocking their
progress. I use an Internet chat program (Slack) to communicate with remote
students and for lower-latency conversations than e-mail.

In their first year, students typically work on a starter project that
builds on a senior student's work, this often becomes a paper with the
new student as a first author.  From there, I work with the student to
craft a trajectory of papers that will form a foundation for the rest
of their graduate studies.  While I'm fairly hands-on with conference
and journal submissions, I never edit proposals or dissertations; I
provide verbal feedback and suggestions but students must find their
own voice in the writing process.  

After graduation, my students have gone on to good positions in
industry (e.g., Viet-An Nguyen, Facebook Data Science) and academia
(e.g., He He, Stanford postdoc; Alvin Grissom II, Assistant Professor
at Ursinus).

\section{Outreach to High School Students}

A major part of my research is making machine learning accessible to
high school students.  My human-computer question answering exhibition
matches have attracted thousands of interested high school students in
\abr{dc}, Chicago, Dallas, and Seattle.

% \section{What Students Say \dots}

% Below are cherry-picked comments from students either sent as an unsolicited
% e-mail (anonymized and presented with permission) or taken from anonymous course
% evaluations.

% \student{ I just started a new job on Monday as a Librarian at [awesome place].
%   I was hired to catalog a special project with the archives and never thought
%   I'd need anything from [introduction to information technology] \dots And then
%   on the first day we started talking about the database of objects.  When we
%   finally got access to the database, I realized it was entirely a SQL database
%   and had to use everything you taught us last semester to help think of how to
%   run the queries and ask the designer what it is capable of.  }


% \student{
% I actually had a job interview for a professional librarian job this week and it
% was about digital libraries.  They asked me questions about what do I know about
% IP validation, HTML, servers, etc.  Thankfully, fresh out of your class, I was
% able to say ``actually, I know a TON about this stuff! There was a question on
% our midterm about IP validations and everything!''

% I wanted to write to let you know that the things I learned in your class this semester are already having a huge impact on my career.
% }

% \student{I showed the \dots app I made for my final project to my boss, and she
%   ran downstairs to [important person]. He came upstairs to see my app and to
%   talk to me about it. Turns out he has a team researching vendors who could
%   provide our library with a mobile presence much like the one I made for my
%   final project in your class. He was excited to see what I'm capable of and
%   will be putting me in touch with the people on the project. This is a big deal
%   for an entry-level librarian like me.}



%%%%%%%%%%%%%%%%%%%%%%%%%%%%%%%%

%\bibliographystyle{../style/acl}
%\bibliography{../bib/journal-full,../bib/jbg}

\end{document}
