%%%%%%%%%%%%%%%%%%%%%%%%%%%%%%%%
%%%
%%% Research Summary
%%%
%%% Author - Steve Hurder
%%%
%%% Date Started: October 12, 2009
%%% Date Completed: November 15 , 2009
%%%%%%%%%%%%%%%%%%%%%%%%%%%%%%%%

\documentclass[11pt]{amsart}
\usepackage{graphicx}
\usepackage{amssymb}
\usepackage{wrapfig}
\usepackage{epstopdf}
\usepackage{nicefrac}

\usepackage[top=1in, bottom=1in, left=1in, right=1in]{geometry}

\newcommand{\student}[1]{\vspace{.5cm}\fbox{\parbox{0.95\linewidth}{{\small #1}}}\vspace{.5cm}}
\newcommand{\abr}[1]{\textsc{#1}}

\begin{document}

 \title{Service Statement}

 \author{Jordan Boyd-Graber}
\address{University of Colorado}
\email{jbg@boydgraber.org}

\date{Fall 2016}


\keywords{}

\maketitle

\section{Professional Service}

In addition to reviewing articles for journals and conferences, I've
served as an area chair for machine learning at \abr{emnlp} 2015,
document classification and topic clustering at \abr{naacl} 2015, and
an ``at large'' area chair at \abr{icml} 2015, helping to shepherd
hundreds of submissions in the one of the fastest growing areas of
these conferences.

I also organized the 2015 \abr{acl} student research workshop with
\abr{nsf} support, which helps introduce students to the research
community, and research-focused workshops at \abr{nips} 2010, 2013,
and \abr{naacl} 2015.

\section{Creating a New Undergraduate Curriculum}

Because I care deeply about undergraduate education, I was the founding chair of
the undergraduate committee for Maryland's College of Information Studies.  This was a
passionate and hard-fought struggle to create a new undergraduate program in a
resource-constrained environment.

I first had to convince elements within the college that an undergraduate
program was indeed a good idea and would not distract or detract from the
College's existing (graduate-only) programs that were already stable,
successful, and well understood.  I crafted a data-driven message that convinced
faculty holdouts and the administration that we should continue a new
undergraduate program that would be well-received by potential students.

The next challenge was resources and coordination.  Because of space constraints,
we had to work with the Universities at Shady Grove and Montgomery County
Community College to make sure we would have sufficient space and students to
launch our new program.  We also had to work with Maryland administration to
have sufficient faculty lines to teach new classes and coordinators to make the
new program a success.

I then created a set of concentrations, courses, and schedules to offer the
courses for initial cohorts.  We drafted course descriptions, learning outcomes,
and worked with feeder programs to work out prerequisites.  Although I left
before it was implemented, Susan Winter led the final implementation of the
program which will soon admit its first students.

\section{Graduate Life}

At both Maryland and Colorado, I worked to improve our recruitment of graduate
students and to make sure they have a rich and rewarding experience once they
come to the university.

At Maryland, I worked to build research depth, research breadth, and cultural
competency in graduate students.  To build research depth, I began the
probabilistic modeling reading group to bring together common users of
probabilistic methods across computer science, linguistics, and information
studies.  To build research breadth, I organized a colloquium series (with
funding from \abr{umiacs}) to bring external speakers working in machine
learning and natural language processing to campus.  To build cultural
competency and improve the social atmosphere of graduate students, I organized a
movie night to introduce international students to the ``standard'' Western
nerdy popular culture (it was also quite popular with \abr{us} students).

At Colorado, I've worked hard as a part of graduate committee to improve the
quality of the graduate students we can recruit.  To boost the crop of
applicants, I traveled extensively to recruit high quality graduate students
from Europe and Asia; obviously this also helped my own recruiting: in my first
to years at Colorado, more applicants expressed interest in working with me than
any other Colorado CS faculty.  As a part of the graduate committee, we
also improved recruiting materials: our brochure, a recruiting video,
interactive demonstrations at visit day, and our offer letter.

\section{Building the Faculty}

At Maryland, I served on the \abr{umiacs} Appointments, Tenure, and Promotions
committee (2012--2013).  At Colorado, I served on the search committee that
hired Jed Brown, Raf Frongillo, and Matt Hammer; with Aaron Clauset, I lead the
Network Science/Machine Learning track that recruited Raf Frongillo.

\section{Outreach to Undergraduates and High Schools}

A major part of my research involves outreach to high schools through computers
that play trivia games.  These events are possible because I've built a
reputation among schools as an effective organizer of events.  In addition to
human-computer tournaments, I've run multiple academic tournaments in Colorado
at both high school and college levels.  Before my arrival, Colorado had not
taken part in this national competition; organizing these events has improved
the state's visibility in academic competitions and brought additional resources
to University of Colorado student organizations (by way of tournament fees).

I've done this in collaboration with groups of passionate
undergraduates.  At Maryland, I advised an existing academic team that
consistently was in the top ten nationally.  At Colorado, I created an
organization from scratch that in 2015 placed second in the Rocky
Mountain region in their first year of competition.

%%%%%%%%%%%%%%%%%%%%%%%%%%%%%%%%

%\bibliographystyle{../style/acl}
%\bibliography{../bib/journal-full,../bib/jbg}

\end{document}
