%%%%%%%%%%%%%%%%%%%%%%%%%%%%%%%%
%%%
%%% Research Summary
%%%
%%% Author - Steve Hurder
%%%
%%% Date Started: October 12, 2009
%%% Date Completed: November 15 , 2009
%%%%%%%%%%%%%%%%%%%%%%%%%%%%%%%%

\documentclass[11pt]{amsart}
\usepackage{graphicx}
\usepackage{amssymb}
\usepackage{wrapfig}
\usepackage{epstopdf}
\usepackage{nicefrac}

\usepackage[top=1in, bottom=1in, left=1in, right=1in]{geometry}

\newcommand{\student}[1]{\vspace{.5cm}\fbox{\parbox{0.95\linewidth}{{\small #1}}}\vspace{.5cm}}

\begin{document}

 \title{Service Statement}

 \author{Jordan Boyd-Graber}
\address{University of Colorado}
\email{jbg@boydgraber.org}

\date{Fall 2016}


\keywords{}

\maketitle

\paragraph{Creating a New Undergraduate Curriculum}

Because I care deeply about undergraduate education, I was the founding chair of
the undergraduate committee for Marylands College of Information.  This was a
passionate and hard-fought struggle to create a new undergraduate program in a
resource-constrained environment.

I first had to convince elements within the college that an undergraduate
program was indeed a good idea and would not distract or detract from the
College's existing (graduate-only) programs that were already stable,
successful, and well understood.  I crafted a data-driven message that convinced
faculty holdouts and the administration that we should continue a new
undergraduate program.

The next challege was resources and coordination.  Because of space constraints,
we had to work with the Universities at Shady Grove and Montgomery County
Community College to make sure we would have sufficient space and students to
launch our new program.  We also had to work with Maryland administration to
have sufficient faculty lines to teach new classes and coordinators to make the
new program a success.

I then created a set of concentrations, courses, and schedules to offer the
courses for initial cohorts.  We drafted course descriptions, learning outcomes,
and worked with feeder programs to work out prerequisites.  Although I left
before it was implemented, Susan Winter led the final implementation of the
program which will soon admit its first students.

\paragraph{Graduate Life}

At both Maryland and Colorado, I worked to improve our recruitment of graduate
students and to make sure they have a rich and rewarding experience once they
come to the university.

At Maryland, I worked to build research depth, research breadth, and cultural
competency in graduate students.

At Colorado, I've worked hard to improve the quality of the graduate students we
can recruit.

\paragraph{Building the Faculty}

At Maryland, I served on the \abr{umiacs} Appointments, Tenure, and Promotions
committee (2012--2013).  At Colorado, I served on the search committe that
hired Jed Brown, Raf Frongillo, and Matt Hammer; with Aaron Clauset, I lead the
Network Science/Machine Learning track that recruited Raf Frongillo.

\paragraph{Outreach to Undergraduate and High Schools}

A major part of my research involves outreach to high schools through computers
that play trivia games.  These events are possible because I've built a
reputation among schools as an effective organizer of events.  In addition to
human-computer tournaments, I've run multiple academic tournaments in Colorado
at both high school and college levels.  Before my arrival, Colorado had not
taken part in this national competition; organizing these events has improved
the state's visibility in academic competitions and brought additional resources
to University of Colorado student organizations (by way of tournament fees).

I've done this in collaboration with groups of passionate undergraduates.  At
Maryland, I advised an existing team that consistently was in the top ten
nationally.  At Colorado, I created an organization from scratch that in 2015
placed second in the Rocky Mountain region in their first year of competition.

%%%%%%%%%%%%%%%%%%%%%%%%%%%%%%%%

%\bibliographystyle{../style/acl}
%\bibliography{../bib/journal-full,../bib/jbg}

\end{document}
