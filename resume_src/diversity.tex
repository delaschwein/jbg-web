%%%%%%%%%%%%%%%%%%%%%%%%%%%%%%%%
%%%
%%% Research Summary
%%%
%%% Author - Steve Hurder
%%%
%%% Date Started: October 12, 2009
%%% Date Completed: November 15 , 2009
%%%%%%%%%%%%%%%%%%%%%%%%%%%%%%%%

\documentclass[11pt]{amsart}
\usepackage{graphicx}
\usepackage{amssymb}
\usepackage{epstopdf}
\usepackage{hyperref}



\usepackage[top=1in, bottom=1in, left=1in, right=1in]{geometry}

\newcommand{\student}[1]{\vspace{.5cm}\fbox{\parbox{0.95\linewidth}{{\small #1}}}\vspace{.5cm}}
\newcommand{\abr}[1]{\textsc{#1}}

\begin{document}

 \title{Statement on Diversity}

 \author{Jordan Boyd-Graber}
\address{University of Maryland}
\email{jbg@boydgraber.org}

\date{Updated January 2023}


\keywords{}

\maketitle

\section{My Background}

My background is fairly privileged: in the lingo, I'm a white, cishet,
relatively able-bodied male who had the background and support to attend an ivy-league graduate school.
%
I was born in the \abr{us}, speak English as my first language, and
other than being raised in a minority religion, exemplify the
demographics that are ``easy mode'' in American society.

Where my background can build connections and empathy with
disadvantaged groups is that I grew up on a relatively poor
background: my mom was the first in her family to graduate college,
and she only did that after I was born.
%
Thus, I observed my mom's struggle as a single mother to work cleaning
motels and navigate the welfare system through her time at community colleges and universities.

The relatively stable life that we had afterward helped illuminate the
value of education and has informed my career in academia.
%
The forgiveness and grace that her teachers and mentors showed her
(i.e., when childcare fell through and she had to bring me to class)
serve as models for how I try to be flexible and understanding to my
own students.
%
And this is also why I prefer to work and teach in public
institutions; the investments that the \abr{us} federal government and
the state of Illinois made in education allowed my mother and myself
to have a better life; private institutions that only teach a handful
of students do not effectively serve all of society.

\section{Mentoring Diverse Students}

And I try to take that broad approach toward recruiting my research
group.
%
Not just for altruistic, egalitarian reasons\dots it helps expose me
to new research topics.
%
Working with a blind student helped motivate our research on new ways
of explaining the output of machine learning models without relying on
graphical visualizations.
%
Working with a hispanic student helped open my eyes to the skewed
demographic assumptions of question answering datasets that mostly
talk about white, American men.
%
Mentoring an African-American student who went on to a faculty
position has been particularly eye opening: both from the advice he needed, his struggles with students, and his insane workload from
being the departments only faculty of color.

This has helped me to realize the importance of advocating for diverse
candidates at the University of Maryland (although not as successfully
as I'd like) and to help build our programs to be more diverse.
%
During grad school, my first advisor was Maria Klawe (until she left
to become president of Harvey Mudd), who helped open my eyes to the
slow attrition of underrepresented groups in the undergraduate
pipeline.
%
In leading the design of the new undergraduate information science major at the University of Maryland, I
specifically tried to avoid these ``weed out'' traps; compared to
computer science, it has substantially better diversity (31\% from
underrepresented groups vs. 13\% for computer science) while also
being close to gender parity.

I've also participated in the Iribe Initiative for Inclusion and Diversity
in Computing at the University of Maryland, which supports, educates,
and mentors students from underrepresented groups in computing fields
at the University of Maryland and supports faculty research to broaden
participation through connecting researchers to projects, hosting
events, and providing support for researchers from underrepresented
groups.
%
Through this program, I served as a mentor in hackathons for
women and minority students at \abr{dc}-area schools.

\section{Knowledge for All}

Beyond the walls of my university, I also feel I have an obligation to
broadly disseminate knowledge: I open source the code to all of my
research projects and ensure that this is written into the contract
language; I post all of my lectures online for anyone to view with
accessible subtitles (which has over a million views); and I regularly
offer remote research opportunities to students.
%
I do this because I know that not everyone has the resources to access
expensive education but that nonetheless almost anyone can benefit
from knowledge.



%%%%%%%%%%%%%%%%%%%%%%%%%%%%%%%%

%\bibliographystyle{../style/acl}
%\bibliography{../bib/journal-full,../bib/jbg}

\end{document}
