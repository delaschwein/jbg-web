\documentclass[11pt,twoside]{article}
\usepackage{alltt}
\usepackage{comment}

\textwidth=6.5in
\oddsidemargin=0.25in
\evensidemargin=0.25in
\topmargin=-0.1in
\footskip=0.8in
\parindent=0.0cm
\parskip=0.3cm
\textheight=8.00in
\setcounter{tocdepth} {3}
\setcounter{secnumdepth} {2}
\sloppy

\newcounter{lecnum}

\newcommand{\answer}[1]{{\bf #1}}

\newcommand{\lecture}[5]{
   \pagestyle{myheadings} \thispagestyle{plain} \newpage
\setcounter{lecnum}{#1} \setcounter{page}{1} \noindent
\begin{center}
\framebox{\vbox{
{Homework #1 \hfill   #2} \\
\vspace{2mm}
{\bf Part of Speech Tagging}\\
\vspace{2mm}
{Due: #3   \hfill #4}\\
\vspace{4mm}}}
\end{center}

\markboth{Homework #1: #3}{Homework #1: #3}\vspace*{4mm}}

\begin{document}

% Add your name in the last bracket if you use this template
\lecture{2}{COS/LIN 280}{October 10, 2008}{Answer Key}

\section{Words and Tokens (8 points)}

Taking a corpus, tokenizer, and stemmer of your choice, find the stem of every word in the corpus.  Answer the following questions:
\begin{enumerate}
\item What corpus, tokenizer, and stemmer did you use?  Did you do anything else to normalize the corpus (remove HTML, convert to lowercase, remove stop words, etc.)?


\answer{For this assignment, we'll just use Sense and Sensibility (which is tokenized by the Gutenberg reader).}


\item How many unique tokens are there?  How many tokens in total are there?


\begin{verbatim}
from nltk.probability import FreqDist
from nltk.corpus import gutenberg
fd = FreqDist()
for word in gutenberg.words(�austen-persuasion.txt�): 
  fd.inc(word)
print "Unique tokens", fd.B()
print "Total tokens", fd.N()
\end{verbatim}


\item How many unique stems are there?


\begin{verbatim}
fd_stem = FreqDist()
from nltk.stem.porter import PorterStemmer
porter = PorterStemmer()
for word in gutenberg.words("austen-persuasion.txt"):
  fd_stem.inc(porter.stem(word))
print "Unique stems", fd_stem.B()
\end{verbatim}


\item What are the ten most common tokens longer than two characters?


\begin{verbatim}
ss = fd.sorted()
count = 0
for word in ss: 
  if len(word) > 2:
    print word, fd[word] 
      count += 1
  if count >= 10:
     break
\end{verbatim}


\item What are the ten most common stems longer than two characters?
(It's fine to filter either the tokens by length or filter the stems
as long as you filter out punctuation.)


\answer{Same as above, but replace ``fd'' with ``fd\_stem.''}


\item What stem has the most unique tokens that got mapped to it?


\begin{verbatim}
from collections import defaultdict
d = defaultdict(dict)
# Create a mapping from each stem to all the unique tokens that got mapped to it
for word in gutenberg.words("austen-persuasion.txt"): 
  d[porter.stem(word.lower())][word] = 1

# Find out the most tokens that got mapped to a single value
max_tokens = max(map(len, d.values()))

# Print out stems that had that many tokens
for stem in d:
  if len(d[stem]) == max_tokens:
    print stem, d[stem].keys()
\end{verbatim}


\end{enumerate}

\section{Tagging and NLTK (18 points)}

\subsection{When taggers go bad (5 points)}

Consider the following sentences:
\begin{enumerate}
\item British Left Waffles on Falkland Islands
\item Teacher Strikes Idle Kids
\item Clinton Wins Budget; More Lies Ahead
\item Juvenile Court to Try Shooting Defendant
\end{enumerate}

(You're also more than welcome to create or find another sentence that
is similarly confusing.)  Choose one of these sentences and tag it in
two different (but plausible) ways.  What sort of taggers, if
any, would get these sentences right?  Which, if any, would get it
wrong?  Back up your answers with either sound reasoning or data.

Note that ``British Left Waffles on Falkland Islands'' can be tagged as follows:
\begin{enumerate}
\item British(Adjective) Left(Noun) Waffles(Verb) on(Preposition) Falkland Islands (Two-word Noun Phrase)
\item British(Noun) Left(Verb) Waffles(Noun) on(Preposition) Falkland Islands (Two-word Noun Phrase)
\end{enumerate}
The first tagging is clearly the correct one, as that meaning corresponds to a historical event, while the meaning of the second tagging is just absurd. 
\\Taggers trained on corpora composed of newspapers would probably tag the sentence correctly, because newspapers often use sentences similar in parts of speech to the correct version to convey similar ideas about the political scene, and rarely ever use sentences tagged similarly to the wrong version.  Indeed, this sentence seems like it was a newspaper headline.
\\Taggers trained on corpora composed of political texts would probably tag the sentence correctly, because in the field of politics, 'left' almosts always denotes a noun and 'waffles' almost always denotes a verb.
\\Taggers trained on any other kind of text would probably tag it wrong, because in other fields, 'left' denotes a verb  more often than a noun, and 'waffles' denotes a noun much more often than a verb.
\\Unigram taggers would probably tag it wrong, because 'waffles' almost always denotes a noun.
\\n-gram taggers would probably tag it correctly, because 'waffles' is almost always a verb in the context of 'Left'.


\subsection{Exploring the tag set (5 points)}
There are 265 distinct words in the Brown Corpus having exactly four possible tags (assuming nothing is done to normalize the word forms).
\begin{enumerate}
\item Create a table with the integers 1\dots 10 in one column, and the number of distinct
words in the corpus having $\{1, \dots, 10\}$ distinct tags.

\begin{center}
\begin{tabular}{l|r}
\hline
Number of Tags & Number of Distinct Words \\
\hline
1 & 47328\\
\hline
2 & 7186\\
\hline
3 & 1146\\
\hline
4 & 265\\
\hline
5 & 87\\
\hline
6 & 27\\
\hline
7 & 12\\
\hline
8 & 1\\
\hline
9 & 1\\
\hline
10 & 2\\
\hline
\end{tabular}
\end{center}


\item For the word with the greatest number of distinct tags, print out sentences from
the corpus containing the word, one for each possible tag.

The word with the greatest number of distinct tags is 'that' with 12 distinct tags.
\\The following are the 12 distinct tags with an example sentence with that tag, one for each possible tag of 'that':
\begin{enumerate}
\item CS-NC: ('When', 'WRB'), ('I', 'PPSS-NC'), ('have', 'HV-NC'), ('instructions', 'NNS-NC'), ('to', 'TO-NC'), ('leave', 'VB-NC'), ('is', 'BEZ'), ('equivalent', 'JJ'), ('in', 'IN'), ('meaning', 'NN'), ('to', 'IN'), ('I', 'PPSS-NC'), ('have', 'HV-NC'), ('instructions', 'NNS-NC'), ('that', 'CS-NC'), ('I', 'PPSS-NC'), ('am', 'BEM-NC'), ('to', 'TO-NC'), ('leave', 'VB-NC'), ('this', 'DT-NC'), ('place', 'NN-NC'), (',', ','), ('dominant', 'JJ'), ('stress', 'NN'), ('is', 'BEZ'), ('ordinarily', 'RB'), ('on', 'IN'), ('leave', 'VB-NC'), ('.', '.')
\item DT-NC: ('Thus', 'RB'), ('to', 'IN-NC'), ('has', 'HVZ'), ('light', 'JJ'), ('stress', 'NN'), ('both', 'ABX'), ('in', 'IN'), ('that', 'DT-NC'), ('was', 'BEDZ-NC'), ('the', 'AT-NC'), ('conclusion', 'NN-NC'), ('that', 'WPO-NC'), ('I', 'PPSS-NC'), ('came', 'VBD-NC'), ('to', 'IN-NC'), ('and', 'CC'), ('in', 'IN'), ('that', 'DT-NC'), ('was', 'BEDZ-NC'), ('the', 'AT-NC'), ('conclusion', 'NN-NC'), ('I', 'PPSS-NC'), ('came', 'VBD-NC'), ('to', 'IN-NC'), ('.', '.')
\item NIL: ('Thus', 'NIL'), (',', ','), ('as', 'NIL'), ('a', 'NIL'), ('development', 'NIL'), ('program', 'NIL'), ('is', 'NIL'), ('being', 'NIL'), ('launched', 'NIL'), (',', ','), ('commitments', 'NIL'), ('and', 'NIL'), ('obligations', 'NIL'), ('must', 'NIL'), ('be', 'NIL'), ('entered', 'NIL'), ('into', 'NIL'), ('in', 'NIL'), ('a', 'NIL'), ('given', 'NIL'), ('year', 'NIL'), ('which', 'NIL'), ('may', 'NIL'), ('exceed', 'NIL'), ('by', 'NIL'), ('twofold', 'NIL'), ('or', 'NIL'), ('threefold', 'NIL'), ('the', 'NIL'), ('expenditures', 'NIL'), ('to', 'NIL'), ('be', 'NIL'), ('made', 'NIL'), ('in', 'NIL'), ('that', 'NIL'), ('year', 'NIL'), ('.', '.')
\item WPS-NC: ('But', 'CC'), ('when', 'WRB'), ('to', 'TO-NC'), ('represents', 'VBZ'), ('to', 'IN-NC'), ('consciousness', 'NN-NC'), ('in', 'IN'), ('that', 'WPS-NC'), ('was', 'BEDZ-NC'), ('the', 'AT-NC'), ('moment', 'NN-NC'), ('that', 'CS-NC'), ('I', 'PPSS-NC'), ('came', 'VBD-NC'), ('to', 'IN-NC'), (',', ','), ('and', 'CC'), ('similarly', 'RB'), ('in', 'IN'), ('that', 'WPS-NC'), ('was', 'BEDZ-NC'), ('the', 'AT-NC'), ('moment', 'NN-NC'), ('I', 'PPSS-NC'), ('came', 'VBD-NC'), ('to', 'IN-NC'), (',', ','), ('there', 'EX'), ('is', 'BEZ'), ('much', 'QL'), ('stronger', 'JJR'), ('stress', 'NN'), ('on', 'IN'), ('to', 'IN-NC'), ('.', '.')
\item CS-HL: ('According', 'IN'), ('to', 'IN'), ('the', 'AT'), ('official', 'JJ'), ('interpretation', 'NN'), ('of', 'IN'), ('the', 'AT'), ('Charter', 'NN-TL'), (',', ','), ('a', 'AT'), ('member', 'NN'), ('cannot', 'MD*'), ('be', 'BE'), ('penalized', 'VBN'), ('by', 'IN'), ('not', '*'), ('having', 'HVG'), ('the', 'AT'), ('right', 'NN'), ('to', 'TO'), ('vote', 'VB'), ('in', 'IN'), ('the', 'AT'), ('General', 'JJ-TL'), ('Assembly', 'NN-TL'), ('for', 'IN'), ('nonpayment', 'NN'), ('of', 'IN'), ('financial', 'JJ'), ('obligations', 'NNS'), ('to', 'IN'), ('the', 'AT'), ('``', '``'), ('special', 'JJ'), ("''", "''"), ('United', 'VBN-TL'), ("Nations'", 'NNS\$-TL'), ('budgets', 'NNS'), (',', ','), ('and', 'CC'), ('of', 'IN'), ('course', 'NN'), ('cannot', 'MD*'), ('be', 'BE'), ('expelled', 'VBN'), ('from', 'IN'), ('the', 'AT'), ('Organization', 'NN-TL'), ('(', '('), ('which', 'WDT'), ('you', 'PPSS'), ('suggested', 'VBD'), ('in', 'IN'), ('your', 'PP\$'), ('editorial', 'NN'), (')', ')'), (',', ','), ('due', 'RB'), ('to', 'IN-HL'), ('the', 'AT-HL'), ('fact', 'NN-HL'), ('that', 'CS-HL'), ('there', 'EX'), ('is', 'BEZ'), ('no', 'AT'), ('provision', 'NN'), ('in', 'IN'), ('the', 'AT'), ('Charter', 'NN-TL'), ('for', 'IN'), ('expulsion', 'NN'), ('.', '.')
\item WPO-NC: ('Thus', 'RB'), ('to', 'IN-NC'), ('has', 'HVZ'), ('light', 'JJ'), ('stress', 'NN'), ('both', 'ABX'), ('in', 'IN'), ('that', 'DT-NC'), ('was', 'BEDZ-NC'), ('the', 'AT-NC'), ('conclusion', 'NN-NC'), ('that', 'WPO-NC'), ('I', 'PPSS-NC'), ('came', 'VBD-NC'), ('to', 'IN-NC'), ('and', 'CC'), ('in', 'IN'), ('that', 'DT-NC'), ('was', 'BEDZ-NC'), ('the', 'AT-NC'), ('conclusion', 'NN-NC'), ('I', 'PPSS-NC'), ('came', 'VBD-NC'), ('to', 'IN-NC'), ('.', '.')
\item WPS: ('She', 'PPS'), ('was', 'BEDZ'), ('a', 'AT'), ('living', 'VBG'), ('doll', 'NN'), ('and', 'CC'), ('no', 'AT'), ('mistake', 'NN'), ('--', '--'), ('the', 'AT'), ('blue-black', 'JJ'), ('bang', 'NN'), (',', ','), ('the', 'AT'), ('wide', 'JJ'), ('cheekbones', 'NNS'), (',', ','), ('olive-flushed', 'JJ'), (',', ','), ('that', 'WPS'), ('betrayed', 'VBD'), ('the', 'AT'), ('Cherokee', 'NP'), ('strain', 'NN'), ('in', 'IN'), ('her', 'PP\$'), ('Midwestern', 'JJ-TL'), ('lineage', 'NN'), (',', ','), ('and', 'CC'), ('the', 'AT'), ('mouth', 'NN'), ('whose', 'WP\$'), ('only', 'AP'), ('fault', 'NN'), (',', ','), ('in', 'IN'), ('the', 'AT'), ("novelist's", 'NN\$'), ('carping', 'VBG'), ('phrase', 'NN'), (',', ','), ('was', 'BEDZ'), ('that', 'CS'), ('the', 'AT'), ('lower', 'JJR'), ('lip', 'NN'), ('was', 'BEDZ'), ('a', 'AT'), ('trifle', 'NN'), ('too', 'QL'), ('voluptuous', 'JJ'), ('.', '.')
\item WPS-HL: ('Withholding', 'VBG-HL'), ('of', 'IN-HL'), ('funds', 'NNS-HL'), ('to', 'IN-HL'), ('schools', 'NNS-HL'), ('that', 'WPS-HL'), ('deny', 'VB-HL'), ('children', 'NNS-HL'), ('on', 'IN-HL'), ('account', 'NN-HL'), ('of', 'IN-HL'), ('race', 'NN-HL'), ('.', '.-HL')
\item WPO: ('It', 'PPS'), ('was', 'BEDZ'), ('nothing', 'PN'), ('that', 'WPO'), ('he', 'PPS'), ('said', 'VBD'), ('or', 'CC'), ('did', 'DOD'), (',', ','), ('but', 'CC'), ('it', 'PPS'), ('seemed', 'VBD'), ('so', 'QL'), ('natural', 'JJ'), ('to', 'IN'), ('her', 'PPO'), ('that', 'CS'), ('she', 'PPS'), ('should', 'MD'), ('be', 'BE'), ('working', 'VBG'), ('for', 'IN'), ('him', 'PPO'), (',', ','), ('looking', 'VBG'), ('forward', 'RB'), ('to', 'IN'), ('his', 'PP\$'), ('eventual', 'JJ'), ('proposal', 'NN'), ('.', '.')
\item CS: ('She', 'PPS'), ('was', 'BEDZ'), ('a', 'AT'), ('living', 'VBG'), ('doll', 'NN'), ('and', 'CC'), ('no', 'AT'), ('mistake', 'NN'), ('--', '--'), ('the', 'AT'), ('blue-black', 'JJ'), ('bang', 'NN'), (',', ','), ('the', 'AT'), ('wide', 'JJ'), ('cheekbones', 'NNS'), (',', ','), ('olive-flushed', 'JJ'), (',', ','), ('that', 'WPS'), ('betrayed', 'VBD'), ('the', 'AT'), ('Cherokee', 'NP'), ('strain', 'NN'), ('in', 'IN'), ('her', 'PP\$'), ('Midwestern', 'JJ-TL'), ('lineage', 'NN'), (',', ','), ('and', 'CC'), ('the', 'AT'), ('mouth', 'NN'), ('whose', 'WP\$'), ('only', 'AP'), ('fault', 'NN'), (',', ','), ('in', 'IN'), ('the', 'AT'), ("novelist's", 'NN\$'), ('carping', 'VBG'), ('phrase', 'NN'), (',', ','), ('was', 'BEDZ'), ('that', 'CS'), ('the', 'AT'), ('lower', 'JJR'), ('lip', 'NN'), ('was', 'BEDZ'), ('a', 'AT'), ('trifle', 'NN'), ('too', 'QL'), ('voluptuous', 'JJ'), ('.', '.')
\item DT: ('``', '``'), ('See', 'VB'), ('that', 'DT'), ('guy', 'NN'), ("''", "''"), ('?', '.'), ('?', '.')
\item QL: ('Then', 'RB'), (',', ','), ('she', 'PPS'), ('was', 'BEDZ'), ('back', 'RB'), ('on', 'IN'), ('her', 'PP\$'), ('feet', 'NNS'), (',', ','), ('winking', 'VBG'), ('and', 'CC'), ('smiling', 'VBG'), ('that', 'QL'), ('enormous', 'JJ'), ('smile', 'NN'), ('(', '('), ('she', 'PPS'), ('had', 'HVD'), ('lots', 'NNS'), ('of', 'IN'), ('wonderful', 'JJ'), ('big', 'JJ'), ('teeth', 'NNS'), ('that', 'CS'), ('you', 'PPSS'), ('never', 'RB'), ('would', 'MD'), ('have', 'HV'), ('suspected', 'VBN'), ('she', 'PPS'), ('had', 'HVD'), ('when', 'WRB'), ('she', 'PPS'), ('was', 'BEDZ'), ('not', '*'), ('smiling', 'VBG'), (')', ')'), ('.', '.')
\end{enumerate}


\end{enumerate}

\subsection{Creating taggers (8 points)}
Create a default tagger and various unigram and n-gram taggers, incorporating backoff,
and train them on part of the Brown corpus.
\begin{enumerate}
\item Create three different combinations of the taggers. Test the accuracy of each
combined tagger on a different part of the Brown corpus. Which combination works best?
The code I used to create the combinations of taggers was:
\begin{verbatim}
t0 = nltk.DefaultTagger('VBZ') 
t1 = nltk.UnigramTagger(brown\_a, backoff=t0)  
comboT1 = nltk.BigramTagger(brown\_a, backoff=t1) 
comboT2 = nltk.TrigramTagger(brown\_a, backoff=t1) 
comboT3 = nltk.TrigramTagger(brown\_a, backoff=comboT1)
\end{verbatim}

The descriptions of each combined tagger are as follows:
\begin{enumerate}
\item comboT1 is a bigram tagger that falls back on a unigram tagger that falls back on a default tagger.
\item comboT2 is a trigram tagger that falls back on a unigram tagger that falls back on a default tagger.
\item comboT3 is a trigram tagger that falls back on a bigram tagger that falls back on a unigram tagger that falls back on a default tagger.
\end{enumerate}
All taggers were trained on category a of the Brown Corpus.
\\
\\Running the combined taggers on category c of the Brown Corpus, yields the following results:
\begin{enumerate}
\item comboT1 has 0.740443199686 accuracy.
\item comboT2 has 0.736979166667 accuracy.
\item comboT3 has 0.741671580189 accuracy.
\end{enumerate}
Therefore, comboT3 works best. 

\item Try varying the size of the training corpus. How does it affect your results?
\\
\\When all taggers were trained on categories a and b of the Brown Corpus, the following results were yielded:
\begin{enumerate}
\item comboT1 has 0.775280070755 accuracy.
\item comboT2 has 0.772602201258 accuracy.
\item comboT3 has 0.776631289308 accuracy.
\end{enumerate}
With a larger training corpus, the accuracies (on category c) of all three combined taggers increase.
\\
\\When all taggers were trained on categories a, b, and e of the Brown Corpus, the following results were yielded:
\begin{enumerate}
\item comboT1 has 0.801051493711 accuracy.
\item comboT2 has 0.799110652516 accuracy.
\item comboT3 has 0.802623820755 accuracy.
\end{enumerate}
With an even larger training corpus, the accuracies (on category c) of all three combined taggers still increase.

\item Create a regexp tagger that does better than the following (make sure you show how you evaluated both taggers):
\end{enumerate}
\begin{verbatim}
>>> patterns = [
... (r'.*ing$', 'VBG'), # gerunds
... (r'.*ed$', 'VBD'), # simple past
... (r'.*es$', 'VBZ'), # 3rd singular present
... (r'.*ould$', 'MD'), # modals
... (r'.*\'s$', 'NN$'), # possessive nouns
... (r'.*s$', 'NNS'), # plural nouns
... (r'^-?[0-9]+(.[0-9]+)?$', 'CD'), # cardinal numbers
... (r'.*', 'NN') # nouns (default)
... ]
\end{verbatim}

\begin{verbatim}
new_patterns=[
(r'^\.$', '.'), #Punctuation marks
(r'^\($', '('),
(r'^\)$', ')'),
(r'^--$', '--'),
(r'^,$', ','),
(r'^:$', ':'),
(r'^n[o\']t$', '*'),
(r'^to$', 'TO'),
(r'^(about|above|across|after|against|along|among|around|at|before|behind|below|beneath|beside|between|beyond|but|by|despite|down|during|except|for|from|in|inside|into|like|near|of|off|on|onto|out|outside|over|past|since|through|throughout|till|to|toward|under|underneath|until|up|upon|with|within|without)$', 'IN'), #prepositions
(r'^([aA]|[tT]he|[nN]o)$', 'AT'), #articles
(r'^(and|or)$', 'CC'), #conjunctions
(r'^[wW](hat|hich)(ever)?$', 'WDT'), #Wh-words
(r'^[wW]hose(ver)?$', 'WP$'),
(r'^[wW]hom$', 'WPO'),
(r'^[wW]ho(so)?(ever)?$', 'WPS'),
(r'^([wW]hen|[wW]ere|[wW]hy|[hH]ow)$', 'WRB'),
(r'^[tT]hat$', 'CS'),
(r'^[hH]ave$', 'HV'), #Have
(r'^[hH]ad$', 'HVD'),
(r'^[hH]aving$', 'HVG'),
(r'^[hH]as$', 'HVZ'),
(r'^[dD]o$', 'DO'), #Do
(r'^[dD]id$', 'DOD'),
(r'^[dD]oes$', 'DOZ'),
(r'^[bB]e$', 'BE'), #Be
(r'^[wW]ere$', 'BED'),
(r'^[wW]as$', 'BEDZ'),
(r'^[bB]eing$', 'BEG'),
(r'^[aA]m$', 'BEM'),
(r'^[bB]een$', 'BEN'),
(r'^[aA]r[et]$', 'BER'),
(r'^[iI]s$', 'BEZ'),
(r'^[tT]his$', 'DT'),
(r'^([sS]ome|[aA]ny)$', 'DTI'),
(r'^[tT]h[eo]se$', 'DTS'),
(r'^.*[eE]ither$', 'DTX'),
(r'.+est$','JJT'), # superlative comparative
(r'.{3,}ly$','RB'), #adverb
(r'^(one|two|three|four|five|six|seven|eight|nine|ten|eleven|twelve|.+teen|twenty|thirty|forty|fifty|sixty|seventy|eighty|ninety|hundred)$','CD'), #numbers
(r'^(first|second|third|fourth|fifth|sixth|seventh|eighth|ninth|tenth|eleventh|twelfth|thirteenth)$','OD'), #ordinals
(r'^[0-9]+(st|nd|rd|th)$','OD'),
(r'.+(thing|one)$','PN'),
(r'.{2,}ing$','VBG'), #consider length
(r'.*[^aeou]ed$','VBD'), #no vowels + ed except for i
(r'.*es$','VBZ'),
(r'.*ould$','MD'),
(r'.*[^s]\'s$','NN$'), #check for no s before the '
(r'.*\'s$','NNS$'), #catch the plurals afterwards
(r'.*[^s]s$','NNS'), #no ss
(r'�-?[0-9]+(.[0-9]+)?$','CD'),
(r'^[A-Z].*[^s]\'s$','NP$'), #proper nouns
(r'^[A-Z].*\'s$','NPS$'),
(r'^[A-Z].*[^s]s$','NPS'),
(r'^[A-Z].*$','NP'),
(r'.*','NN') ] #catchall

With this I get: 0.6304961827353569 accuracy
compared to the original tagger: 0.19460649490003576

I tested using the Brown corpus, categories A and B like this:
brown_sents = nltk.corpus.brown.tagged_sents( categories=['a','b'] )
nltk.tag.accuracy(regexp_tagger, brown_sents)
\end{verbatim}

\section{Viterbi Algorithm (14 Points)}

Consider the following sentences written in Klingon.  For each
sentence, the part of speech of each ``word'' has been given (for ease
of translation, some prefixes/suffixes have been treated as words),
along with a translation.  Using these training sentences, we're going
to build a hidden Markov model to predict the part of speech of an
unknown sentence using the Viterbi algorithm.

\begin{tabular}{llllll}
  N  & PRO  & V   & N         & PRO  \\
pa'Daq   & ghah & taH & tera'ngan & 'e   \\
room (inside)    & he   & is  & human     & of   \\ 
\multicolumn{6}{c}{{\em The human is in the room}} \\
\end{tabular}

\begin{tabular}{llll}
V   & N      & V    & N              \\
ja'chuqmeH & rojHom & neH  & tera'ngan        \\
in order to parley  & truce  & want & human \\
\multicolumn{4}{c}{{\em The enemy commander wants a truce in order to parley}}
\end{tabular}

\begin{tabular}{lllllll}
N         & V      & N     & CONJ & N         & V    & N \\
tera'ngan & qIp   & puq   & 'eg  & puq & qIp  & tera'ngan \\
human     & bit  & child & and  & child & bit & child \\
\multicolumn{7}{c}{{\em The child bit the human, and the human bit the child}}
\end{tabular}

\subsection{Emission Probability (4 points)}

Compute the frequencies of each part of speech in the table below for
nouns and verbs.  We'll use a smoothing factor of 0.1 (as discussed in
class) to make sure that no event is impossible; add this number to
all of your observations.  Two parts of speech have already been
done for you.  After you've done this, compute the emission probabilities in a similar table.

\begin{center}
\begin{tabular}{|l|c|c|c|c|c|}
\hline
   & NOUN & VERB & CONJ & PRO \\
\hline
'e &  0.1   &  0.1   &  0.1    &  1.1   \\
\hline
'eg &  0.1   &  0.1   &  1.1    &  0.1 \\
\hline
ghaH &  0.1   & 0.1   &  0.1    &  1.1  \\
\hline
ja'chuqmeH &  0.1   &  1.1   &  0.1    &  0.1   \\
\hline
legh &  0.1   &  0.1   &  0.1    &  0.1 \\
\hline
neH &  0.1   &  1.1   &  0.1    &  0.1   \\
\hline
pa'Daq &  1.1   &  0.1   &  0.1    &  0.1 \\
\hline
puq &  2.1   &  0.1   &  0.1    &  0.1 \\
\hline
qIp &  0.1   &  2.1   &  0.1    &  0.1 \\
\hline
rojHom &  1.1   &  0.1   &  0.1    &  0.1\\
\hline
taH &  0.1   &  1.1   &  0.1    &  0.1 \\
\hline
tera'ngan &  4.1   &  0.1   &  0.1    &  0.1 \\
\hline
yaS &  0.1   &  0.1   &  0.1    &  0.1   \\
\hline
\end{tabular}
\end{center}

\subsection{Start and Transition Probability (4 points)}

Now, for each part of speech, total the number of times it
transitioned to each other part of speech.  Again, use a smoothing
factor of 0.1.  After you've done this, compute the start and transition probabilities.

\begin{center}
\begin{tabular}{|c|c|c|c|c|c|}
\hline
      & NOUN & VERB & CONJ & PRO \\
\hline
START &  &   &  &   \\
\hline
N     &   &   & 1.1  & 2.1  \\
\hline
V     &   &   & 0.1  & 0.1  \\
\hline
CONJ  &   &   &  0.1 & 0.1 \\
\hline
PRO   &  &   &  0.1 & 0.1 \\
\hline
\end{tabular}
\end{center}

\subsection{Viterbi Decoding (6 points)}

Now consider the following sentence: ``tera'ngan legh yaS''.  

\begin{enumerate}
\item Suppose that we knew ``legh'' were a pronoun (it's not, but pronouns often act like its true part of speech in Klingon).  What would be the probability of each of the four parts of speech for ``yaS''?
\\
\\The probability ``yaS'' is a 
\begin{description}
\item[noun] is $.1/1.4 \approx 7\%$
\item[verb] is $1.1/1.4 \approx 79\%$
\item[conjunction] is $.1/1.4 \approx 7\%$
\item[pronoun] is $.1/1.4 \approx 7\%$
\end{description}

\item Create the decoding matrix of this sentence for nouns and verbs (ignore other parts of speech if you want; doing so won't prevent you from finding the right answer).  You should have at least six numbers: $\log \delta_{n}(k)$ for $n=1\dots3$ and $k$ for both nouns and verbs.
\\
\\For $k=$N:
\begin{itemize}
\item $\delta_1(k)=\pi_k\beta_{k,x_1} = (2.1/3.4)(4.1/9.3) = .272$
\\$\log\delta_1(k) = \log .272 = -1.30$ 
\item $\delta_1(k)\theta_{k,k} = .272(.1/6.4) = .00425$
\\$\delta_1(j)\theta_{j,k} = .00514(5.1/5.4) = .00485$
\\$\delta_2(k)= max_j(\delta_1(j)\theta_{j,k})\beta_{k,x_2} = .00485(.1/9.3) = .0000522$
\\$\log\delta_2(k) = \log .0000522 = -9.86$ 
\item $\delta_2(k)\theta_{k,k} = .0000522(.1/6.4) = .000000816$
\\$\delta_2(j)\theta_{j,k} = .00210(5.1/5.4) = .00198$
\\$\delta_3(k)= max_j(\delta_1(j)\theta_{j,k})\beta_{k,x_2} = .00198(.1/9.3) = .0000213$
\\$\log\delta_3(k) = \log .0000213 = -10.8$ 
\end{itemize}
For $k=$V:
\begin{itemize}
\item $\delta_1(k) = \pi_k\beta_{k,x_1} = (1.1/3.4)(.1/6.3) = .00514$
\\$\log\delta_1(k) = \log .00514 = -5.27$ 
\item $\delta_1(k)\theta_{k,k} = .00514(.1/5.4) = .0000952$
\\$\delta_1(j)\theta_{j,k} = .272(3.1/6.4) = .132$
\\$\delta_2(k)= max_j(\delta_1(j)\theta_{j,k})\beta_{k,x_2} = .132(.1/6.3) = .00210$
\\$\log\delta_2(k) = \log .00210 = -6.17$ 
\item $\delta_2(k)\theta_{k,k} = .00210(.1/5.4) = .0000389$
\\$\delta_2(j)\theta_{j,k} = .0000522(3.1/6.4) = .0000253$
\\$\delta_3(k)= max_j(\delta_1(j)\theta_{j,k})\beta_{k,x_2} = .0000389(.1/6.3) = .000000617$
\\$\log\delta_3(k) = \log .000000617 = -14.3$ 
\end{itemize}
\begin{center}
\begin{tabular}{|c|c|c|c|c|c|}
\hline
POS   & $\log\delta_1(k)$ &$\log\delta_2(k)$ & $b_2$ & $\log\delta_3(k)$ & $b_3$ \\
\hline
N     & $-1.30$ & $-9.86$	& V & $-10.8$ & V \\
\hline
V     & $-5.27$ & $-6.17$ & N & $-14.3$ & V \\
\hline
Word  & tera'ngan  &\multicolumn{2}{c}{legh} \vline  &\multicolumn{2}{c}{yaS} \vline \\
\hline
\end{tabular}
\end{center}

\item What is the most likely sequence of parts of speech?
\\
\\Noun, verb, noun
\\because $-10.8>-14.3$, so it most likely ends in a noun, which is most likely preceded by a verb, which is most likely preceded by a noun.

\item What is the probability of your previous answer?
\\
\\The probability of my previous answer is $\delta_3(noun)= .0000213$.

\item (For fun, not for credit) What do you think this sentence means?  What word is the subject of the sentence?

It seems that Klingon has OVS verb order.  So the subject of the sentence is ``yaS,'' which means officer.  ``legh'' is a verb that means ``to see,'' so that sentence means ``the officer sees the human.''

\end{enumerate}



\end{document}