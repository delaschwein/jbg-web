\documentclass[11pt,twoside]{article}
\usepackage{alltt}
\usepackage{comment}

\textwidth=6.5in
\oddsidemargin=0.25in
\evensidemargin=0.25in
\topmargin=-0.1in
\footskip=0.8in
\parindent=0.0cm
\parskip=0.3cm
\textheight=8.00in
\setcounter{tocdepth} {3}
\setcounter{secnumdepth} {2}
\sloppy

\newcounter{lecnum}

\newcommand{\answer}[1]{{\bf #1}}

\newcommand{\lecture}[5]{
   \pagestyle{myheadings} \thispagestyle{plain} \newpage
\setcounter{lecnum}{#1} \setcounter{page}{1} \noindent
\begin{center}
\framebox{\vbox{
{Homework #1 \hfill   #2} \\
\vspace{2mm}
{\bf Parsing}\\
\vspace{2mm}
{Due: #3   \hfill #4}\\
\vspace{4mm}}}
\end{center}

\markboth{Homework #1: #3}{Homework #1: #3}\vspace*{4mm}}


\begin{document}

% Add your name in the last bracket if you use this template
\lecture{6}{CMSC 723}{October 28, 2013}{}


\section{Context-Free Grammars (30 points)}

Consider the following context free grammar:

\begin{center}
\begin{tabular}{lc}
Intermediate Rules 	& Terminal Rules \\
\hline
S $\rightarrow$ NP VP &  D $\rightarrow$ a $|$ the \\
NP $\rightarrow$ (D) NOM & V $\rightarrow$ slept $|$ disappeared $|$ loved $|$ gave $|$ relied \\
 VP $\rightarrow$ V (NP) (NP) & N $\rightarrow$ cat $|$ dog $|$ roof $|$ man \\
  NOM $\rightarrow$ N  &  P $\rightarrow$ in $|$ on $|$ with \\
  NOM  $\rightarrow$ NOM PP & CONJ $\rightarrow$ and $|$ or \\
   VP  $\rightarrow$  VP PP &  \\
    PP  $\rightarrow$  P NP &  \\
    X   $\rightarrow$  X CONJ X &  \\
\end{tabular}
\end{center}

We designate the start state to be S, and X in the last rule stands for any part of speech (thus the last rule allows NP  $\rightarrow$  NP CONJ NP but not NP$\rightarrow$  NP CONJ VP).  

\begin{enumerate}
\item Give a grammatical English sentence that has one an only one valid interpretation in this grammar.  Draw the tree corresponding to it.
\item Give a grammatical English sentence that has more than one valid interpretation in this grammar.  Draw two trees, and discuss whether the meaning in English is ambiguous.
\item Give a sentence that has a valid interpretation in this grammar but is not grammatical.
\begin{comment}
The dog slept the man.
\end{comment}
\item Give a grammatical English sentence using only these terminals that does not have a valid interpretation in this grammar.  Why doesn't it have a valid interpretation?
\begin{comment}
And the man slept.
With the dog the man slept.
\end{comment}
\item Can you give an upper bound for the number of unique sequences of terminals this grammar admits?  If not, why not?
\end{enumerate}

\section{Penn Treebank (30 points)}

In this section, let's consider the 10\% sample of the Penn Treebank included in NLTK.  

\begin{enumerate}
\item What is the minimum and maximum height of sentences in the Treebank?  Give an example tree for both.  How does the depth correlate with sentence length?
%\item Write a python function called ``subjects'' to identify all of the subjects in a Peen Treebank sentence (this includes subjects of all clauses, not just independent clauses).  Submit this file separately as ``oitusername-subjects.py''.  For example, your function should work like this:
%\begin{verbatim}
%>>> t = nltk.corpus.treebank.parsed_sents()[0]
%>>> for ii in subjects(t): print ii
%... 
%(NP-SBJ
%  (NP (NNP Pierre) (NNP Vinken))
%  (, ,)
%  (ADJP (NP (CD 61) (NNS years)) (JJ old))
%  (, ,))
%\end{verbatim}
\item In the following sentence:
\begin{verbatim}
( (S 
    (NP-SBJ-1 
      (NP (NNP Rudolph) (NNP Agnew) )
      (, ,) 
      (UCP 
        (ADJP 
          (NP (CD 55) (NNS years) )
          (JJ old) )
        (CC and) 
        (NP 
          (NP (JJ former) (NN chairman) )
          (PP (IN of) 
            (NP (NNP Consolidated) (NNP Gold) (NNP Fields) (NNP PLC) ))))
      (, ,) )
    (VP (VBD was) 
      (VP (VBN named) 
        (S 
          (NP-SBJ (-NONE- *-1) )
          (NP-PRD 
            (NP (DT a) (JJ nonexecutive) (NN director) )
            (PP (IN of) 
              (NP (DT this) (JJ British) (JJ industrial) (NN conglomerate) ))))))
    (. .) ))
\end{verbatim}
what does ``(-NONE- *-1)'' mean?  Explain both in terms of this sentence specifically and what it means in general linguistically.

\item In order to build a PCFG, we need to estimate the probability of all of the production rules.  What is the probability of each of the following production rules?  This is a rare case where we actually want the MLE estimate, as zeros rule out possible parse trees.  How many unique non-terminal productions are there?  How about terminal productions?
\begin{center}
\begin{tabular}{c|c}
Production Rule & Probability \\
\hline
NP $\rightarrow$ DT JJ NNS & \\
VP $\rightarrow$ VB NP & \\
ADJP $\rightarrow$ JJ PP & \\
VP $\rightarrow$ MD VP & \\
NN $\rightarrow$ 'stock' & \\
IN $\rightarrow$ 'like' & \\
IN $\rightarrow$ 'on' & \\
IN $\rightarrow$ 'with' & \\
IN $\rightarrow$ 'about' & \\
IN $\rightarrow$ 'over' & \\
\end{tabular}
\end{center}

\end{enumerate}

\section{Probabilistic Context-Free Grammars (40 points)}

Consider the following sentence and the following PCFG:

\begin{center}
	{\em time flies like an arrow }\\ 
\begin{tabular}{cl|cl}
\multicolumn{2}{c}{Intermediate Rules} & \multicolumn{2}{c}{Terminal Rules} \\
\hline
S $\rightarrow$ NP VP& 0.8 & N $\rightarrow$ time& 0.5 \\
S $\rightarrow$ VP& 0.2&  N $\rightarrow$ flies& 0.3 \\
VP $\rightarrow$ V NP& 0.5 & N $\rightarrow$ arrow& 0.2\\ 
VP $\rightarrow$ V PP& 0.3&  V $\rightarrow$ time& 0.3 \\
VP $\rightarrow$ VP PP& 0.2 & V $\rightarrow$ flies& 0.3 \\
NP $\rightarrow$ Det N& 0.3&  V $\rightarrow$ like& 0.4 \\
NP $\rightarrow$ N& 0.3&  P $\rightarrow$ like &1.0 \\
NP $\rightarrow$ N N& 0.2 & Det $\rightarrow$ an & 1.0 \\
NP $\rightarrow$ NP PP& 0.2 & \\
PP $\rightarrow$ P NP & 1.0 & \\
\end{tabular}
\end{center}

\begin{enumerate}
\item What are the four possible parse trees for this sentence?
\item Draw a CKY chart parse for the sentence and determine the most likely parse of this sentence.  Show your work.
\end{enumerate}


\end{document}