\documentclass[11pt, amssymb, a4paper, one column]{article}
\usepackage{times, amsmath, amssymb, cancel, changepage, url, gensymb, graphicx, mathrsfs, amsthm, lipsum, fancyhdr}
\usepackage[margin=.75in]{geometry}

\newcommand{\ihat}{\mathbf {\hat \imath}}
\newcommand{\jhat}{\mathbf {\hat \jmath}}



%you only really need a documentclass/ packages to make a document, the following is just nice formatting. You can easily remove it or look up something else
%that you like better

\pagestyle{fancy}
\lhead{Jordan Boyd-Graber, University of Maryland}
\rhead{\large Natural Language Processing Shouldn't be a Black Box \normalsize}
%note how I use "\#" to denote "#." This is because # is reserved (for something). Similarly, % is reserved for comments, so to use it you need \%.
%other such things include { }.

\fancypagestyle{plain}{}

%end of header, beginning of document

\begin{document}

Machine learning is ubiquitous: detecting spam e-mails, flagging fraudulent
purchases, and providing the next movie in a Netflix binge.  But few users at
the mercy of machine learning \emph{outputs} know what's happening behind the
curtain.  My research goal is to demystify the black box for non-experts by
creating \emph{algorithms that can inform, collaborate with, compete with, and
  understand users} in real-world settings.

This is at odds with mainstream machine learning---take topic models.  Topic
models are sold as a tool for understanding large data collections: lawyers
scouring Enron e-mails for a smoking gun, journalists making sense of Wikileaks,
or humanists characterizing the oeuvre of Lope de Vega.  But topic models'
proponents never asked what those lawyers, journalists, or humanists
needed. Instead, they optimized \emph{held-out likelihood}. When my colleagues
and I developed the \emph{interpretability} measure to assess whether topic
models' users understood their outputs, we found that interpretability and
held-out likelihood were negatively correlated! The topic
modeling community (including me) had fetishized complexity at the expense of
usability.

Since this humbling discovery, I've built topic models that are a collaboration
between humans and computers.  The computer starts by proposing an organization
of the data.  The user responds by separating confusing clusters, joining
similar clusters together, or comparing notes with another
user.  The model updates and then directs the user to
problematic areas that it knows are wrong.  This is a huge improvement over the
``take it or leave it'' philosophy of most machine learning algorithms.

This is not only a technical improvement but also an improvement to the social
process of machine learning adoption. A program manager who used topic models to
characterize \textsc{nih} investments uncovered interesting synergies and
trends, but the results were unpresentable because of a fatal flaw: one of the
700 clusters lumped urology together with the nervous system, anathema to
\textsc{nih} insiders. Our tools allow non-experts to fix such
obvious (to a human) problems, allowing machine learning
algorithms to overcome the \emph{social} barriers that often hamper adoption.

% Fix ``on the fly''

Our realization that humans have a lot to teach machines led us to
\emph{simultaneous machine interpretation}. Because verbs end phrases
in many languages, such as German and Japanese, existing algorithms
must wait until the end of a sentence to begin translating (since
English sentences have verbs near the start). We learned tricks from
professional human interpreters---passivizing sentences and guessing
the verb---to translate sentences sooner, letting
speakers and algorithms cooperate together and enabling more natural
cross-cultural communication.

The reverse of cooperation is competition; it also has much to teach
computers. I've increasingly looked at language-based games whose clear goals
and intrinsic fun speed research progress.  For example, in \emph{Diplomacy},
users chat with each other while marshaling armies for world conquest.
Alliances are fluid: friends are betrayed and enemies embraced as the game
develops. However, users' conversations hint when friendships break:
betrayers writing ostensibly friendly messages become more
polite, stop talking about the future, and change how much they
write.  Diplomacy may be a nerdy game, but it is a fruitful
testbed to teach computers to understand messy, emotional human interactions.

A game with higher stakes is politics.  However, just like Diplomacy,
the words that people use reveal their underlying goals; computational
methods can help expose the ``moves'' political players can use.  With
collaborators in political science, we've built models that: show when
politicians in debates strategically change the topic to influence
others; frame topics to reflect political leanings; use subtle
linguistic phrasing to express their political leaning; or create
political subgroups with larger political movements.

Conversely, games also teach humans \emph{how computers think}.  Our
trivia-playing robot played four former Jeopardy champions
in front of 600 high school students. The computer's early
lead evaporated because we foolishly projected the computer's thought process for
all to see.  Our opponents learned to read the algorithm's ranked dot
products and adjusted their strategy. In five years
of teaching machine learning, students never so
quickly learned how linear classifiers work.  The probing questions from
high school students in the audience showed they understood too.
(Later, we played again against Ken Jennings; he sat in front of
the dot products and our system won.)

Advancing machine learning requires closer, more natural interactions.
However, we still require much of the user---reading distributions or
dot products---rather than natural interactions.  Document
exploration tools should describe in words what a cluster is, not just
provide inscrutable word clouds.  Deception detection systems should
say \emph{why} a betrayal is imminent.  Question answers should
explain \emph{how} it knows Aaron Burr shot Alexander Hamilton: thus
helping human players of trivia games either as a study partner or as
a teammate at a competition. My work will complement machine
learning's ubiquity with transparent, empathetic, and useful
interactions with users.

\end{document}
