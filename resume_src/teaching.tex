%%%%%%%%%%%%%%%%%%%%%%%%%%%%%%%%
%%%
%%% Research Summary
%%%
%%% Author - Steve Hurder
%%%
%%% Date Started: October 12, 2009
%%% Date Completed: November 15 , 2009
%%%%%%%%%%%%%%%%%%%%%%%%%%%%%%%%

\documentclass[11pt]{amsart}
% Copyright (c) 2012 Cies Breijs
%
% The MIT License
%
% Permission is hereby granted, free of charge, to any person obtaining a copy
% of this software and associated documentation files (the "Software"), to deal
% in the Software without restriction, including without limitation the rights
% to use, copy, modify, merge, publish, distribute, sublicense, and/or sell
% copies of the Software, and to permit persons to whom the Software is
% furnished to do so, subject to the following conditions:
%
% The above copyright notice and this permission notice shall be included in
% all copies or substantial portions of the Software.
%
% THE SOFTWARE IS PROVIDED "AS IS", WITHOUT WARRANTY OF ANY KIND, EXPRESS OR
% IMPLIED, INCLUDING BUT NOT LIMITED TO THE WARRANTIES OF MERCHANTABILITY,
% FITNESS FOR A PARTICULAR PURPOSE AND NONINFRINGEMENT. IN NO EVENT SHALL THE
% AUTHORS OR COPYRIGHT HOLDERS BE LIABLE FOR ANY CLAIM, DAMAGES OR OTHER
% LIABILITY, WHETHER IN AN ACTION OF CONTRACT, TORT OR OTHERWISE, ARISING FROM,
% OUT OF OR IN CONNECTION WITH THE SOFTWARE OR THE USE OR OTHER DEALINGS IN THE
% SOFTWARE.

%%% LOAD AND SETUP PACKAGES

\usepackage[margin=0.5in]{geometry} % Adjusts the margins

\usepackage{multicol} % Required for multiple columns of text

\usepackage{mdwlist} % Required to fine tune lists with a inline headings and indented content

\usepackage{relsize} % Required for the \textscale command for custom small caps text

\usepackage[pdftex]{hyperref} % Required for customizing links
\usepackage[dvipsnames]{xcolor} % Required for specifying custom colors
\definecolor{dark-blue}{rgb}{0.15,0.15,0.4} % Defines the dark blue color used for links
\hypersetup{colorlinks,linkcolor={dark-blue},citecolor={dark-blue},urlcolor={dark-blue}} % Assigns the dark blue color to all links in the template

\usepackage{tgpagella} % Use the TeX Gyre Pagella font throughout the document
\usepackage[T1]{fontenc}
\usepackage{microtype} % Slightly tweaks character and word spacings for better typography

\pagestyle{empty} % Stop page numbering

%----------------------------------------------------------------------------------------
%	DEFINE STRUCTURAL COMMANDS
%----------------------------------------------------------------------------------------

\newenvironment{indentsection} % Defines the indentsection environment which indents text in sections titles
{\begin{list}{}{\setlength{\leftmargin}{\newparindent}\setlength{\parsep}{0pt}\setlength{\parskip}{0pt}\setlength{\itemsep}{0pt}\setlength{\topsep}{0pt}}}{\end{list}}

\newcommand*\maintitle[2]{\noindent{\LARGE \textbf{#1}}\ \ \ \emph{#2}\vspace{0.3em}} % Main title (name) with date of birth or subtitle

\newcommand*\roottitle[1]{\subsection*{#1}\vspace{-0.3em}\nopagebreak[4]} % Top level sections in the template

\newcommand{\headedsection}[3]{\nopagebreak[4]\begin{indentsection}\item[]\textscale{1.1}{#1}\hfill#2#3\end{indentsection}\nopagebreak[4]} % Section title used for a new employer

\newcommand{\headedsubsection}[3]{\nopagebreak[4]\begin{indentsection}\item[]\textbf{#1}\hfill\emph{#2}#3\end{indentsection}\nopagebreak[4]} % Section title used for a new position

\newcommand{\bodytext}[1]{\nopagebreak[4]\begin{indentsection}\item[]#1\end{indentsection}\pagebreak[2]} % Body text (indented)

\newcommand{\inlineheadsection}[2]{\begin{basedescript}{\setlength{\leftmargin}{\doubleparindent}}\item[\hspace{\newparindent}\textbf{#1}]#2\end{basedescript}\vspace{-1.7em}} % Section title where body text starts immediately after the title

\newcommand*\acr[1]{\textscale{.85}{#1}} % Custom acronyms command

\newcommand*\bull{\ \ \raisebox{-0.365em}[-1em][-1em]{\textscale{4}{$\cdot$}} \ } % Custom bullet point for separating content

\newlength{\newparindent} % It seems not to work when simply using \parindent...
\addtolength{\newparindent}{\parindent}

\newlength{\doubleparindent} % A double \parindent...
\addtolength{\doubleparindent}{\parindent}

\newcommand{\breakvspace}[1]{\pagebreak[2]\vspace{#1}\pagebreak[2]} % A custom vspace command with custom before and after spacing lengths
\newcommand{\nobreakvspace}[1]{\nopagebreak[4]\vspace{#1}\nopagebreak[4]} % A custom vspace command with custom before and after spacing lengths that do not break the page

\newcommand{\spacedhrule}[2]{\breakvspace{#1}\hrule\nobreakvspace{#2}} % Defines a horizontal line with some vertical space before and after it


\newcommand{\grant}[7]{

\headedsubsection{#1}{#2--#3 (#4)} { % title, start date, end date, source
	\inlineheadsection{Investigators:}{#5}
	\vspace{1.2em}
	\inlineheadsection{Award:}{\$#6 (Share: \$#7)}
	\vspace{1.2em}
}
}

\newcommand{\colabgrant}[8]{

\headedsubsection{#1}{#2--#3 (#4)} { % title, start date, end date, source
	\inlineheadsection{Investigators:}{#5}
	\vspace{1.2em}
	\inlineheadsection{Award:}{\$#6 (Share: \$#7)}
	\vspace{1.2em}
	\inlineheadsection{Collaboration:}{#8}
	\vspace{1.2em}
}
}

\usepackage{graphicx}
\usepackage{amssymb}
\usepackage{epstopdf}
\usepackage{mdwlist}
\usepackage{hyperref}

%\usepackage[top=1in, bottom=1in, left=1in, right=1in]{geometry}

\newif\ifumd\umdfalse
\newcommand{\umdtext}[2]{
\ifumd
#1 #2
\else
#2
\fi
}

\newcommand{\student}[1]{\vspace{.5cm}\fbox{\parbox{0.95\linewidth}{{\small #1}}}\vspace{.5cm}}
\newcommand{\abr}[1]{\textsc{#1}}

\begin{document}

 \title{Teaching Statement}

 \author{Jordan Boyd-Graber}
\address{University of Maryland}
\email{jbg@boydgraber.org}

\date{Fall 2020}


\keywords{}

\maketitle

\headedsection{\umdtext{III.A. }{Courses Taught}}{}{

\headedsubsection{CMSC 470: Natural Language Processing}{\textsc{umd}, Spring 2019}{
  \bodytext{40 students, First Offering of Course with Permanent Number}}


\headedsubsection{CMSC 723: Computational Linguistics I}{\textsc{umd}, Fall 2018}{
  \bodytext{60 students}}


\headedsubsection{CMSC 389A: Practical Deep Learning}{\textsc{umd}, Spring 2018}{
  \bodytext{30 students, Student-led course initiative: classroom instruction by Sujith Vishwajith}}

\headedsubsection{INST 414: Data Science Methods}{\textsc{umd}, Spring 2018}{
  \bodytext{50 students}}


\headedsubsection{CMSC 726: Machine Learning}{\textsc{umd}, Fall 2017}{
  \bodytext{60 students}}


\headedsubsection{CSCI 7000: Advanced Machine Learning for Natural
  Language Processing}{Colorado, Spring 2017}{
  \bodytext{24 students}}


\headedsubsection{CSCI 3022: Introduction to Data Science Algorithms}{Colorado, Fall 2016}{
	\bodytext{100 students}}

\headedsubsection{CSCI 5622: Machine Learning}{Colorado, Fall 2015}{
	\bodytext{104 students}
}

\headedsubsection{CSCI 5622: Machine Learning}{Colorado, Spring 2015}{
	\bodytext{58 students}
}

\headedsubsection{CSCI/LING 5832: Natural Language Processing}{Colorado, Fall 2014}{
	\bodytext{32 students}
}

\headedsubsection{INST 737: Digging into Data}{\textsc{umd}, Spring 2014}{
	\bodytext{29 students}}

\headedsubsection{CMSC/LING 723 / INST 735: Computational Linguistics I} {\textsc{umd}, Fall 2013}{
	\bodytext{45 students}
	}

\headedsubsection{LING 848B / CMSC 828B: Bayesian Nonparametrics} {\textsc{umd}, Spring 2013}{
	\bodytext{15 students}}

\headedsubsection{INST 737: Digging into Data} {\textsc{umd}, Spring 2013}{
	\bodytext{30 students}}

\headedsubsection{LBSC 690: Introduction to Information Technology} {\textsc{umd}, Fall 2012}{
	\bodytext{30 students}}

\headedsubsection{INST728C / CMSC 773 / LING 773: Computational Linguistics II} {\textsc{umd}, Spring 2012}{
	\bodytext{11 Students}}

\headedsubsection{LBSC 690: Introduction to Information Technology} {\textsc{umd}, Fall 2011}{
	\bodytext{30 students}}

\headedsubsection{INFM 718G: Web Scale Information Processing Applications} {\textsc{umd}, Spring 2011}{
	\bodytext{12 students}}

\headedsubsection{LBSC 690: Introduction to Information Technology} {\textsc{umd}, Fall 2010}{
	\bodytext{ 30 students}}

\headedsubsection{COS/LIN 280: Computational Linguistics} {Princeton, Fall 2008}{
	\bodytext{40 students, Taught by Christiane Fellbaum (I developed homeworks)}}

}


\section{Experience}

Teaching is one of the primary reasons I am in academia; in particular, I want
to excite students in science and technology. Here I describe how
participation and course activities help build this excitement in students and
how these methods help students build a career.



\subsection{Interactive Classrooms}

I find traditional lecture dull, even when I'm the person in front of
the class.  Thus, I usually ``flip'' my classrooms to record
lectures so students can watch at home at their leisure.  This moves
discussion, working on homework, and student questions into class time
(where before they happened at home or not at all).

Beyond structuring class to allow for many questions and discussion
points, I end most classes with a difficult discussion of a key
concept discussed in class and encouraging the class to work
collaboratively together to arrive at a solution.  For example:
\begin{itemize*}
  \item After introducing relational databases, I work together with the class to
design a database scheme to serve the needs of a library circulation
system.  We talk through suggestions on what data should be stored,
how it should be represented, and what implications those choices
have.
  \item When teaching classifiers, students walk through the classification
    algorithms on tiny datasets (for example, documents with one or two words).
  \item When teaching annotation frameworks (called coding guides in the
    social sciences), I ask students to annotate data and
    compute their inter-annotator agreement.  I have discovered that this, more
    than any collections of examples I can provide, effectively convinces
    students the importance of having good input data for their algorithms.
\end{itemize*}

For technical classes with a large hands-on component, I provide
students with ample opportunities to explore resources in a supportive
environment where I or their classmates---working together in small
groups---can help them overcome the small, unexpected hurdles that can
appear while exploring new concepts.

I also strive to encourage interaction outside of the classroom.  Every class I
teach uses a virtual space (Piazza) to encourage discussion and mutual support.  This
helps the entire class get answers to common questions, and it also serves as a
catalyst for spontaneous, unexpected communications: students sharing useful
resources with each other, organizing study groups, or sharing sci-fi books that
illustrate concepts.

I also try to be accessible to students through a variety of methods (e.g.,
Piazza, e-mail, office hours) to quickly answer questions and address their
concerns so that they do not get frustrated and so that they can make progress.

\subsection{Evaluation and Course Activities}

I typically structure classes with a few small, practical assignments (typically
three to five), a midterm, and a course project.

The assignments give continuity to the class and allow students to practice
skills introduced in the class; I encourage students to work together to solve
homework problems.  For example, in a course introducing students to information
technology, one assignment is to create a basic webpage; for a more advanced
class introducing students to cloud computing, one assignment is to create an
algorithm to play the ``Kevin Bacon'' game (i.e., to find the shortest path
between an actor and Kevin Bacon based on co-starring roles).

The midterm serves as a reality check for both my students and me.  I design
exams with five to six questions (of which students must answer a subset) that
synthesize disparate concepts from the course in a problem context (e.g., for a
machine learning course, creating a set of features for a new classification
problem).  Based on the results of the midterm, I can identify students that
might need extra help or what areas I need cover in more detail.

Finally, I use group projects to ensure students can apply what
they've learned in the course.  It reinforces key concepts from the
course, connects those concepts to the rest of their curriculum and
research, and often serves as a launching point to things that are
useful in the real world.  Projects in my courses have become a comedy
troupe's website, have unearthed previously unknown primary sources on
local history, helped students advance in the workplace, and resulted
in academic publications.  As a testament to the effectiveness of
these relationships, after one graduate course I taught (Computational
Linguistics II, UMD CMSC 773), I ended up publishing papers with four
of the eight students.

The projects help individuals calibrate the course to their own needs
and abilities; because the project is directed to things they
care about, the teams working on the projects typically stretch their
abilities more than I could through predefined assignments.  It also
helps build the close connection with the subject area, incubating a
comfort with science and technology.

\subsection{Mentoring Undergraduates}

Students realize that I'm passionate about research and that I'm also
approachable so that they can explore their interests. I've worked with nine
undergraduate students, our work with undergraduates has been published in top
venues like \abr{naacl} and \abr{chi}, and the undergraduates I've worked with
have gone on to have successful research careers (e.g., Lester Mackey became
faculty at Stanford before moving to Microsoft Research, Eric Wallace is now a
PhD student at Berkeley).

\subsection{Mentoring Graduate Students}

I've graduated ten PhD students and am the chair or co-chair for
eight current students.  I like to have a group meeting every other
week with students I'm working with (broadly construed), and
one-on-one meetings as needed with students, typically once a week. In
addition, everyone in my group (me included) sends a weekly e-mail to
the group saying: what they worked on that week; what they plan to work
on next week; anything that's holding them up or blocking their
progress. I use an Internet chat program (Slack) to communicate with remote
students and for lower-latency conversations than e-mail.

In their first year, students typically work on a starter project that
builds on a senior student's work, this often becomes a paper with the
new student as a first author.  From there, I work with the student to
craft a trajectory of papers that will form a foundation for the rest
of their graduate studies.  While I'm fairly hands-on with conference
and journal submissions, I never edit proposals or dissertations; I
provide verbal feedback and suggestions but students must find their
own voice in the writing process.

After graduation, my students have gone on to good positions in industry (e.g.,
Viet-An Nguyen, Facebook Data Science) and academia (e.g., He He, NYU faculty;
Mohit Iyyer, UMass faculty; Alvin Grissom II, Assistant Professor at
Haverford).

\subsection{Outreach to High School Students}

A major part of my research is making machine learning accessible to
high school students.  My human-computer question answering exhibition
matches have attracted thousands of interested high school students in
\abr{dc}, Chicago, Dallas, and Seattle.

\subsection{Pedagogical development projects}

From 2011--2013, I chaired \abr{umd}'s Information Science undergraduate
education committee. During this time, we developed a new undergraduate degree
program to complement the university's existing strengths (and exploding
enrollments) in computer science. I developed the initial proposal,
articulation agreements with local community colleges, planned course
offerings, and shepherded the proposal through college and provost approval.
Susan Winter and Vedat Diker (and many others) took over the actual
implementation of the program, which became the university's fastest growing
major (958 majors in less than five years of existance).

\section{Practice and Reflection}

In my decade of teaching, my teaching style has evolved substantially. The
biggest change has gone from traditional synchronous lectures to asynchronous
recorded videos, which started in 2013. When I went back to lecturing
synchronously, I found the experience excruciating\dots I had no idea who the
students were, how they were doing, and getting them to ask questions was
nearly impossible. The biggest benefit of asynchronous lecture was
communicating often with students and seeing throughout the semester how
students were absorbing and applying the material.

\subsection{Evaluations of Teaching}

Both the computer science department and the iSchool at the University of Maryland have yearly reviews of teaching.
At the University of Colorado, I participated in the Classroom Learning Interview Process, which uses a structured interview of students in the classroom to access faculty teaching.

Based on the feedback from these evaluations, in the last fiew years, I have developed new strategies to:
\begin{itemize*}
  \item Break online video lectures into segments of no more than fifteen minutes
  \item Have more, smaller homeworks (when teaching support is sufficient)
  \item Call on tables rather than individuals to spark initial conversations in flipped classrooms
  \item Provide context to questions when answering them (e.g., ``I suspect Kim is asking this question because\dots'')
\end{itemize*}

\section{Example Sylabuses}

Part of my committment to public education is to ensure that all of my course
materials (lectures, syllabus, assignments) are publicly accessible. The most
recent version of each of my courses is listed at
\url{http://users.umiacs.umd.edu/~jbg/static/courses.html}.  My lectures, which have approximately half a million views, are available on YouTube at \url{https://www.youtube.com/c/JordanBoydGraber/videos}.

% \section{What Students Say \dots}

% Below are cherry-picked comments from students either sent as an unsolicited
% e-mail (anonymized and presented with permission) or taken from anonymous course
% evaluations.

% \student{ I just started a new job on Monday as a Librarian at [awesome place].
%   I was hired to catalog a special project with the archives and never thought
%   I'd need anything from [introduction to information technology] \dots And then
%   on the first day we started talking about the database of objects.  When we
%   finally got access to the database, I realized it was entirely a SQL database
%   and had to use everything you taught us last semester to help think of how to
%   run the queries and ask the designer what it is capable of.  }


% \student{
% I actually had a job interview for a professional librarian job this week and it
% was about digital libraries.  They asked me questions about what do I know about
% IP validation, HTML, servers, etc.  Thankfully, fresh out of your class, I was
% able to say ``actually, I know a TON about this stuff! There was a question on
% our midterm about IP validations and everything!''

% I wanted to write to let you know that the things I learned in your class this semester are already having a huge impact on my career.
% }

% \student{I showed the \dots app I made for my final project to my boss, and she
%   ran downstairs to [important person]. He came upstairs to see my app and to
%   talk to me about it. Turns out he has a team researching vendors who could
%   provide our library with a mobile presence much like the one I made for my
%   final project in your class. He was excited to see what I'm capable of and
%   will be putting me in touch with the people on the project. This is a big deal
%   for an entry-level librarian like me.}



%%%%%%%%%%%%%%%%%%%%%%%%%%%%%%%%

%\bibliographystyle{../style/acl}
%\bibliography{../bib/journal-full,../bib/jbg}

\end{document}
